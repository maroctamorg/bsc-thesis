\begin{proof}
	We first show that a frame spans the algebra. Consider an arbitrary set of vectors expressed in a given orthogonal basis $\{v_i \equiv \sum_{j=1}^n v_i^je_j\}_{i=1}^m$; we expand their product using bilinearity:
	\[\prod_{i=1}^m v_i = \prod_{i=1}^m \sum_{j=1}^n v_i^j e_j = \sum_{j_i : \{1..m\} \to \{1..n\}} \left(\prod_{i=1}^m v_i^{j_i}\right) \left(\prod_{i=1}^m e_{j_i}\right)\]
	It follows from Corollary \ref{c:orthonormal-bases} that the product of the basis vectors in each term is a $k$-blade, where $k$ is the number of unit vectors appearing an odd number of times in the product. The above shows that the product of vectors is indeed a linear combination of elements in the frame generated by the orthogonal basis above.

	A frame can be mapped into the power set of a basis as it constitutes a subset of all the possible combinations of the basis elements without repetition and up to permutation (we identify the unit scalar in the frame with the empty subset of the basis). As such, the maximal number of distinct elements of a frame is $2^n$.

	Since the algebra has dimension $2^n$, the frame must have $2^n$ linearly independent elements and indeed constitute a linear basis for the algebra.
\end{proof}
