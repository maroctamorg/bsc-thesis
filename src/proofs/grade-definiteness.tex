\begin{proof}
	First, we prove that for a given orthogonal basis, the grade of a blade is unique, then we show that the grade of a blade is independent of the choice of basis. The result extends directly to general multivectors.

	By Theorem \ref{t:frame}, a blade is simply an element of some frame of $\G$: since a frame is a linear basis for the Geometric Algebra, it follows that frame elements of different grade are linearly independent. The grade is thus unique w.r.t. to a specific choice of frame.


	Let $\{e_j\}_{j=1}^n$ and \[w_i = \sum_{j=1}^n A_{ij} e_j\] be two orthonormal bases (orthogonality suffices, we choose this purely for notational convenience).
	

	The orthogonality of $\{w_i\}_{i=1}^k$ implies that for $a \neq b$:
		
	\[w_a|w_b = (\sum_{j=1}^n A_{aj} e_j)|(\sum_{r=1}^n A_{br} e_r) = \sum_{j=1}^n A_{aj}A_{bj} = 0\]

	Expanding the full product, we have:
	\[w_1...w_k = (\sum_{j_1=1}^n A_{1{j_1}}e_{j_1})...(\sum_{j_k=1}^n A_{k{j_k}}e_{j_k})\]

	We can index the terms in the above product as follows: at each of the $k$ factors choose one of the terms in the sum (i.e. $A_{it} e_t$), construct a sequence $t(i)$ out of subsequent choices.

	The terms with grades lesser than $k$ (w.r.t. the orthonormal basis) are those terms for which at least one choice is repeated, we can use this to index all possible such terms: given a choice of $(q, m) \in \{1..k\}^2$ where the repetition ought to occur, we are free to choose any $c \in \{1..n\}$ for which $t(q) = t(m) = c$, moreover we are free to vary all other terms independently so we must sum over all subsequences $F(q,m) \equiv \{f(i): \{1..k\}\setminus\{q,m\} \to \{1..n\}\}$. 

	Thus, we may collect all lower grade terms in the following sum:
	\begin{align*}
		&\sum_{(q,m)} \sum_{f \in F} \sum_{c=1}^n (-1)^{\mg{m-q-1}}(A_{1f(1)}...A_{qc}A_{mc}...A_{kf(k)})e_{f(1)}...e_{f(k)} \\
	&= \sum_{(q,m)} \sum_{f \in F} (-1)^{\mg{m-q-1}}(A_{1f(1)}...A_{kf(k)})e_{f(1)}...e_{f(k)} \sum_{c=1}^n (A_{qc}A_{mc}) = 0
	\end{align*}
	
	So all lower-grade terms vanish. That the total product is non-zero simply follows from the fact that the algebra would otherwise permit a frame of dimension smaller than $2^n$.
\end{proof}
