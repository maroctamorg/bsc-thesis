\begin{definition}{Elements of $\G$}
    \begin{block}{Identities}
        As a unitary algebra, $\G$ has unique additive and multiplicative identities: $0, 1 \in \R \subset \G$
    \end{block}
    \begin{block}{Scalars}
        We refer to reals $a \in \R \subset \G$ as scalars, or $\textit{grade-0}$ vectors.
    \end{block}
    \begin{block}{1-vectors}
        Elements $v \in \V \subset \G$ are called 1-vectors, or simply vectors.
	\begin{block}{k-blades} \label{d:k-blades}
        Products $\bv{e_1}\bv{e_2}...\bv{e_k}$ of k anticommuting vectors; we say that a k-blade has grade k. These are also called simple k-vectors.
    \end{block}
    \begin{block}{k-versors}
        Arbitrary products $\bv{v_1}\bv{v_2}...\bv{v_k}$ of k vectors.
    \end{block}
    \begin{block}{Multivectors}
		Finite sums of versors. By Axiom (2), every element $V \in \G$ is a multivector. We say $A$ is a \textbf{homogeneous} multivector iff it is a sum of k-blades for a given $k \in \N$; otherwise we say $A$ is of mixed grade.
    \end{block}
\end{definition}
