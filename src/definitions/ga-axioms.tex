\begin{axiom}\label{a:ga-axiom1}
	$\G$ contains $\R$ as a subalgebra and $\V$ as a subspace; these sets are \emph{disjoint} and \emph{generate} the entire algebra: that is to say, all elements of the algebra are obtained by multiplying and adding elements from the two sets. We call elements of $\R$ \emph{scalars}, and elements of $\V$ \emph{vectors}.
\end{axiom}

\begin{axiom}\label{a:ga-axiom2}
	The formal product of a scalar and a vector corresponds with the \emph{multiplication by a scalar of the vector space}.
\end{axiom}

\begin{axiom}\label{a:ga-axiom3}
    The square of a vector corresponds to its inner product with itself; we denote the inner product on $\V$ as $u|v$.
	\[\forall v \neq 0 \in \V \quad v^2 \equiv vv = v|v = |v|^2\]
\end{axiom}

%\begin{axiom}\label{a:ga-axiom4}
%	The formal product on $\V$ is positive-definite, i.e.:
%\[\forall v \neq 0 \in \V \quad vv > 0\]
%\end{axiom}

\begin{remark}\label{r:ga-axiom3}
	The above axiom distinguishes a Clifford Algebra from a Geometric Algebra in our treatment: a Clifford Algebra only requires a quadratic form on a vector space $\V$. Standard treatments of Geometric Algebra also do not make such a strict requirement and thus the terms are often used interchangeably: we will not be doing so in this thesis.
\end{remark}

\begin{axiom}\label{a:freest}
	The product of orthogonal vectors is a formal product subject only to bilinearity and antisymmetry; products over distinct subsets of a given orthonormal basis are linearly independent and together with the unit scalar form a vector space-basis for the algebra.
\end{axiom}

%\begin{axiom}\label{a:ga-axiom5}
%%	The antisymmetrized product of linearly independent vectors	produces an element which does not belong to $\R \bigoplus \V$, we call this product the \textbf{exterior product} and denote it $u \wedge v = \frac{1}{2} (uv - vu)$. 
%	The product of linearly independent vectors produces an element that is not in the linear span of its factors: i.e. if $\{v_i\}_{i=1}^n$ is a linearly independent set, then:
%	\[\prod_{i=1}^n v_i \notin \mathrm{span}\{\prod_{j=1}^n v_{i_j} \forall i_j : \{1..n\} \to \{0..n\} \} \equiv F(\{v_i\}_{i=1}^n)\]
%	were we define $v_0 = 1$ for notational convenience.
%\end{axiom}
