\begin{definition}[Identity Elements]
    As a unital algebra, $\G(\V)$ has unique additive and multiplicative identity elements: $0, 1 \in \R \subset \G(\V)$
\end{definition}
\begin{definition}[Scalars]
    We refer to reals $a \in \R \subset \G(\V)$ as scalars, or $\textit{grade-0}$ vectors.
\end{definition}
\begin{definition}[1-vectors]
    Elements $v \in \V \subset \G(\V)$ are called 1-vectors, or simply vectors.
\end{definition}
\begin{definition}[k-blades] \label{d:k-blades}
	Products $e_{i_1}e_{i_2} \ldots e_{i_k}$ of orthogonal vectors; we say that a k-blade has grade k. These are also called simple k-vectors.
\end{definition}
\begin{definition}[Versors]
Arbitrary products $\bv{v}_1\bv{v}_2 \ldots \bv{v}_k$ of vectors.
\end{definition}
\begin{definition}[Multivectors]
	Finite sums of versors. By Axiom \ref{a:ga-axiom1}, every element $V \in \G(\V)$ is a multivector.
\end{definition}
