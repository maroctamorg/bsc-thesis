\begin{definition}[Identities]
    As a unitary algebra, $\G(\V)$ has unique additive and multiplicative identities: $0, 1 \in \R \subset \G(\V)$
\end{definition}
\begin{definition}[Scalars]
    We refer to reals $a \in \R \subset \G(\V)$ as scalars, or $\textit{grade-0}$ vectors.
\end{definition}
\begin{definition}[1-vectors]
    Elements $v \in \V \subset \G(\V)$ are called 1-vectors, or simply vectors.
\end{definition}
\begin{definition}[k-blades]
    Products $\bv{e_1}\bv{e_2}...\bv{e_k}$ of k anticommuting vectors; we say that a k-blade has grade k.
\end{definition}
\begin{definition}[k-versors]
    Arbitrary products $\bv{v_1}\bv{v_2}...\bv{v_k}$ of k vectors.
\end{definition}
\begin{definition}[Multivectors]
	Finite sums of versors. By Axiom \ref{a:ga-axiom1}, every element $V \in \G(\V)$ is a multivector. We say $A$ is a \textbf{homogeneous} (or simple) multivector iff it is a sum of k-blades for a given $k \in \N$; otherwise we say $A$ is of mixed grade.
		We introduce the grade operator $\g{A}{k}$ which returns the k-grade component of $A$.
		The grade of a multivector $\gr{A} \in \N$ is the grade of the maximum-grade term in $A$.
\end{definition}
