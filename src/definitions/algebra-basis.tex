\begin{definition}[Basis]\label{d:algebra-basis}
	A subset $B$ of an algebra $A$ is said to be a \textbf{basis} for the algebra $A$ iff it is the basis of $A$ as a vector space.
	If such a set exists, we say that the algebra is \textbf{free}. We say that the algebra $A$ is \textbf{finite-dimensional} iff it has a finite basis. 
\end{definition}
\begin{remark}
	Later on, we will introduce the distinct concept of the basis of a geometric algebra - to differentiate the two we will sometimes refer to the above definition of basis as the linear (or vector-space) basis of the algebra, and refer to the linear basis of a geometric algebra as a frame (this is standard terminology that will be introduced later). It should be clear in any case that whenever we refer to the basis of an algebra that is not a geometric algebra, we mean the above definition.
\end{remark}
