%\section{The Geometric Family of Algebras}\label{s:uga}
%
%In the standard formalism, the fundamental concept in Geometric Algebra is that of the \textbf{Universal Geometric Algebra} (UGA for short): an infinite-dimensional abstract algebra obeying a certain set of axioms, within which all the Geometric Algebras are contained (as subalgebras) \cite{ga-origin}. 
%
%For our purposes, whereby we will limit ourselves to geometric algebras over finite-dimensional inner product spaces, I have found this to be unnecessary and not the best suited approach. Instead we shall formulate as our starting point, a family of geometric algebras:
%\begin{definition}[Geometric Family of Algebras]
%	The \textbf{Geometric Family of Algebras} is a family of algebras obeying a specific set of axioms. Its elements (the Geometric Algebras) are specified by the choice of a finite-dimensional inner product space over the real numbers.
%\end{definition}
%
%A \textbf{GFA} is then a template for Geometric Algebras: given a finite-dimensional linear space, and an inner product, there exists a unique axiom-abiding algebra which contains the linear space and whose symmetrized product corresponds to the prescribed inner product.
%
%We will, however, discuss the generally applicable definitions and results in the context of an abstract algebra whose formal product obeys the axioms, and whose abstract linear space will always be assumed to be of a sufficiently high (finite) dimension such that it does not get in the way of the argument.
%
%\newpage

\section{Axioms and Definitions}\label{s:axioms-definitions}

The Geometric Algebra $\G$ of a finite-dimensional inner product space $\V$ is the unital associative algebra over the reals which obeys the following axioms:
\begin{axiom}\label{a:ga-axiom1}
    $\G$ contains $\R$ as a subalgebra and $\V$ as a subspace; these generate the entire algebra. We call elements of $\R$ scalars, and elements of $\V$ vectors.
\end{axiom}

\begin{axiom}\label{a:ga-axiom2}
	The formal product of a scalar and a vector corresponds with the multiplication by a scalar of the vector space.
\end{axiom}

\begin{axiom}\label{a:ga-axiom3}
    The square of a vector corresponds to its inner product with itself.
	\[\forall v \neq 0 \in \V \quad v^2 \equiv vv = v|v = |v|^2\]
\end{axiom}

%\begin{axiom}\label{a:ga-axiom4}
%	The formal product on $\V$ is positive-definite, i.e.:
%\[\forall v \neq 0 \in \V \quad vv > 0\]
%\end{axiom}

\begin{remark}\label{r:ga-axiom3}
	The above axiom distinguishes a Clifford Algebra from a Geometric Algebra in our treatment: a Clifford Algebra only requires a quadratic form on a vector space $\V$. Standard treatments of Geometric Algebra also do not make such a strict requirement and thus the terms are often used interchangeably: we will not be doing so in this thesis.
\end{remark}

%\begin{axiom}\label{a:ga-axiom5}
%%	The antisymmetrized product of linearly independent vectors	produces an element which does not belong to $\R \bigoplus \V$, we call this product the \textbf{exterior product} and denote it $u \wedge v = \frac{1}{2} (uv - vu)$. 
%	The product of linearly independent vectors produces an element that is not in the linear span of its factors: i.e. if $\{v_i\}_{i=1}^n$ is a linearly independent set, then:
%	\[\prod_{i=1}^n v_i \notin \mathrm{span}\{\prod_{j=1}^n v_{i_j} \forall i_j : \{1..n\} \to \{0..n\} \} \equiv F(\{v_i\}_{i=1}^n)\]
%	were we define $v_0 = 1$ for notational convenience.
%\end{axiom}

We refer to the product of such an algebra as a \textbf{geometric product}.

%\begin{remark}
%	That the algebra is the 'freest' simply means that we specify the product just enough that it satisfies the axioms, and impose no further relations. One could for example define a geometric product on $\R^3$ as the sum of the inner and cross products: the generated algebra would satisfy the axioms, however it would not be the freest since the cross product imposes the additional equivalence relation:
%	\[e_ie_{[i+1]_3} \equiv e_{[i+2]_3} \quad \forall i \in \{1,2,3\}\]
%	where $[a]_3$ is a shorthand notation for $a$ mod $3$.
%	A modern, abstract formulation of the concept has this property naturally since the Geometric Algebra is constructed from more abstract objects by imposing only the required relations: I have not taken this route given the scope and target audience of the thesis. %\textcolor{red}{FETCH A REFERENCE?!}
%\end{remark}

It follows immediately from Axiom \ref{a:ga-axiom3}:
\begin{lemma}\label{l:invertibility}
	All vectors are invertible with respect to the geometric product and the inverse is given by:
	\[v^{-1} = \frac{v}{\mg{v}^2}\]
\end{lemma}



%\begin{lemma}[Inner Product]\label{l:inner-product}
	The symmetrized product of two vectors, \[s(u,v) = \frac{1}{2}(uv + vu)\] corresponds to the inner product on $\V$, i.e.: \[s(u,v) = u|v \forall u,v \in \V\]
\end{lemma}

%\begin{proof}
	Consider the following expression, and recall Axiom \ref{a:ga-axiom3}:
    \begin{align*}
        &(u+v)^2 = u^2 + v^2 + uv + vu \Leftrightarrow \\
        &2u|v = (u+v)^2 - u^2 - v^2 \in \R
    \end{align*}
	So the symmetrized product maps into $\R$ (i.e. $| : \V \times \V \to \R$). By Definition \ref{d:algebra}, the geometric product is bilinear and so is the symmetrized product as it is a linear function of the bilinear products; moreover, by Axiom \ref{a:ga-axiom4}, it is also positive-definite. We conclude that $u|v = uv + vu$ is indeed an inner product on $\V$.
	
\end{proof}


%We have seen that $\R \bigoplus \V$ is closed under the symmetric part of the product; the algebra must thus be generated by the antisymmetric part of the product:
%\begin{definition}[Exterior Product]\label{d:exterior-product}
	The linear rejection $R(\prod_{i=1}^n, F(\{v_i\}_{i=1}^n))$ of a product with regards to the linear span of its factors is denoted
	\[ \bigwedge_{i=1}^n v_i \equiv v_1 \wedge ... \wedge v_n \]
	We call this exterior (or outer) product.
\end{definition}


%We have observed that the axioms place a restriction on the symmetric part of the product between two vectors. The antisymmetric part, on the other hand, we denote $u \wedge v$ and call it the exterior product. This product is specified by Axiom \ref{a:freest}, as presented in the following lemma:
%\input{src/lemmas/outer-product}
%\input{src/proofs/outer-product}
%bears no restriction: since the Geometric Algebra is the most general algebra obeying the axioms, we identify this part of the product with a formal bilinear and antisymmetric product, we call it the \textbf{exterior} product.
One of the main results of this section is the principal property of the geometric product.
\begin{theorem}[Geometric Product of Vectors]\label{t:geometric-product}
	The geometric product between two vectors $u$ and $v$ can be written as the sum of the \textbf{inner product} $u|v$ of $\V$ and a bilinear antisymmetric product which we denote $u\wedge v$ and call \textbf{exterior product}.
\[uv = u|v + u\wedge v\] 
\end{theorem}

\begin{proof}
	Let $u$ and $v \in \V$, we may expand the two vectors in an orthogonal basis $\{e_i\}_{i=1}^n$ and separate their product into a symmetric and antisymmetric part using bilinearity:
	\begin{align*}
		uv &= \left(\sum_i \alpha_ie_i \right)\left(\sum_j \beta_je_j \right) \\
		   &= \sum_{i} (\alpha_i\beta_i)e_ie_i + \sum_{i \neq j} (\alpha_i\beta_j)e_ie_j \\
		   &= u|v + u\wedge v
	\end{align*}
	Where in the last step we have made the following observations:
	\begin{enumerate}
		\item since the product $e_ie_i$ corresponds to the inner product $e_i|e_i$ by Axiom \ref{a:ga-axiom3}, it follows directly that the symmetric part of the sum is the inner product of $u$ and $v$ in $\V$.

		\item the antisymmetric sum is a linear combination of products of orthogonal vectors; by Axiom \ref{a:freest}, we obtain that it is indeed a bilinear antisymmetric product and identify this with the exterior product $u\wedge v$.
	\end{enumerate}
\end{proof}


As a corollary, we obtain the following self-evident propositions.
%\begin{corollary}
	The exterior product of a linearly dependent set of vectors is zero.
\end{corollary}

%\begin{proof}
	PROBLEM: exterior product is currently defined in terms of geometric product only for two vectors...
	Let $\{v_i\}_{i=1}^n$ be a linearly dependent set of vectors, i.e.:
	\[\exists \{\alpha_i\}_{i=1}^{n-1} : v_n \equiv \sum_{i=1}^{n-1} \alpha_i v_i\]
	Then, consider their outer product and recall bilinearity and antisymmetry:
	\begin{align*}
		\bigwedge_{i=1}^n v_i &= v_1 \wedge ... \wedge \sum{i=1}^{n-1} \alpha_i v_i \\
		&= (\bigwedge_{i=1}^{n-1} v_i) \
\end{proof}

\begin{corollary}
	A pair of vectors are orthogonal if and only if they anticommute with respect to the geometric product.
\end{corollary}


\begin{corollary}\label{c:orthonormal-bases}
	An orthonormal basis $\{e_i\}_{i=1}^n$ obeys the following relations:
	\begin{align*}
		e_ie_j &= -e_je_i \\
		e_ie_i &= 1
	\end{align*}
\end{corollary}


Before we make some final remarks about existence, uniqueness and construction, we ought to familiarise ourselves with some specific terminology.
\begin{definition}
	We refer to reals $a \in \R \subset \G(\V)$ as \textbf{scalars}, or $\textit{grade-0}$ vectors.
\end{definition}
\begin{definition}
	Elements $v \in \V \subset \G(\V)$ are called \textbf{$1$-vectors}, or simply vectors.
\end{definition}
\begin{definition} \label{d:k-blades}
	A \textbf{$k$-blade} is a product $e_{i_1}e_{i_2} \ldots e_{i_k}$ of orthogonal vectors; these are also called simple $k$-vectors.
\end{definition}
\begin{definition}
	A \textbf{versor} is an arbitrary product $v_1v_2 \ldots v_k$ of vectors.
\end{definition}
\begin{definition}
	\textbf{Multivectors} are finite sums of versors. By Axiom \ref{a:generation}, every element $V \in \G(\V)$ is a multivector.
\end{definition}


A critical lemma to which we will often make implicit reference is the following:
\begin{lemma}\label{l:ga-expansion}
	Every element $V \in \G(\V)$ has a unique decomposition (with respect to a given basis) as a sum of k-blades.
\end{lemma}

\begin{proof}
	Every multivector can be written as a finite sum of versors, so it suffices to prove the statement for an arbitrary versor.
	Consider an arbitrary set of vectors expressed in a given orthonormal basis $\{v_i \equiv \sum_{j=1}^n v_i^je_j\}_{i=1}^m$; we expand their product using bilinearity:
	\[\prod_{i=1}^m v_i = \prod_{i=1}^m \sum_{j=1}^n v_i^j e_j = \sum_{j_i : \{1..m\} \to \{1..n\}} \left(\prod_{i=1}^m v_i^{j_i}\right) \left(\prod_{i=1}^m e_{j_i}\right)\]
	It follows from Corollary \ref{c:orthonormal-bases} that the product of the basis vectors in each term is a $k$-blade, where $k$ is the number of unit vectors appearing an odd number of times in the product.
	Uniqueness follows from the freeness of the algebra: the product of orthogonal vectors is a formal product subject only to bilinearity and antisymmetry, so that distinct products of orthonormal basis vectors are linearly independent and form in fact a vector space-basis for the whole algebra.
\end{proof}


We are now justified in presenting the following definitions:
\begin{definition}[Grade]\label{d:grade}
	We introduce the grade operator $\g{A}{k}$ which returns the $k$-grade component of $A$.
	The grade of a multivector $\gr{A} \in \N$ is the grade of the maximum-grade term in $A$.
	We say $A$ is a \textbf{homogeneous} multivector of grade $k$ iff it is a sum of $k$-blades for a given $k \in \N$ (when we wish to make it explicit, we denote it $A_k$); otherwise we say $A$ is of mixed grade.
\end{definition}

\begin{definition}[Basis]\label{d:basis}
	The \textbf{basis} of a Geometric Algebra $\G(\V)$ is simply the basis of the generating inner-product space $\V$.
\end{definition}

\begin{definition}
	A \textbf{frame} of a Geometric Algebra $\G(\V)$ is the set of distinct (up to permutation) $k$-blades formed from an orthogonal basis of $\G(\V)$.
\end{definition}

\begin{definition}
	The \textbf{pseudoscalar} of a geometric algebra $\G(\V)$ is the highest grade element in its frame. It is unique up to scalar multiplication and including permutation of factors.
	The square of a pseudoscalar is a scalar: by convention, it is normalized such that it squares to $1$ or $-1$ (defining the orientation of the frame).
\end{definition}


The distinction between basis and frame is very important: a frame is to a basis, as a geometric algebra is to its generating vector space.
For finite-dimensional spaces, we can always produce an orthonormal basis by the Gram-Schmidt process, so we will henceforth assume all bases to be orthonormal unless otherwise specified.

A frame can be identified one-to-one with the power set of a basis, as it is composed of all the possible combinations of the basis elements without repetition and up to permutation (we identify the unit scalar in the frame with the empty subset of the basis). It follows that the dimension of a geometric algebra $\G$ (as a vector space) is $2^{\gr{\G}}$.

It is then clear that for every inner product space there exists a unique Geometric Algebra, and the recipe for its construction is quite simple:
\begin{enumerate}
	\item construct an orthonormal basis for the inner-product space: this will be the basis of the algebra
	\item construct the frame of the algebra by taking products of basis vectors
	\item generate the rest of the algebra as a vector space with the frame as its basis
\end{enumerate}

% With the above terminology we can also straightforwardly prove the Universal Property of a Geometric Algebra.
% \begin{definition}[Universal Property of a Geometric Algebra]
	Let $\V$ be a finite-dimensional vector space over the reals, and let $j: \V \times \V \to \R$ be an inner product, then there exists unique geometric algebra $\G(\V)$ and canonical injection $i: \V \to \G(\V)$ such that the following diagram commutes:
	\textcolor{red}{NOT QUITE!}
\end{definition}

% \begin{definition}[Universal Property of a Geometric Algebra]
	Let $\V$ be a finite-dimensional vector space over the reals, and let $j: \V \times \V \to \R$ be an inner product, then there exists unique geometric algebra $\G(\V)$ and canonical injection $i: \V \to \G(\V)$ such that the following diagram commutes:
	\textcolor{red}{NOT QUITE!}
\end{definition}


A brief note on conventions: here on out, whenever left unspecified, the precedence of products is inner$\to$wedge$\to$geometric.

%\newpage
%
%\section{Construction: Bases and Frames}\label{s:bases-frames}
%
%A Geometric Algebra is only realized into a concrete algebra once an inner-product space is specified. It is then straightforward to construct the whole algebra given a basis for the inner-product space. But first, we need some specific terminology.


%\newpage

\section{Products}\label{s:products}
We will now go on to consider general expressions and properties of products between arbitrary multivectors.
First off, we generalize the definition of the inner and outer products to homogeneous multivectors.
\begin{definition}\label{d:inner-product1}
	The inner product $A_r|B_s$ between two homogeneous multivectors $A_r$ and $B_s$ is the lowest-possible grade term in their product. We will later see that this is the $\mg{r-s}$-grade term; if the product does not have such a term, then we say that the inner product is 0.
\end{definition}

\begin{definition}\label{d:outer-product1}
	The outer product $A_r \wedge B_s$ between two homogeneous multivectors $A_r$ and $B_s$ is the highest-possible grade term in their product. We will later see that this is the $\mg{r+s}$-grade term; if the product does not have such a term, then we say that the outer product is 0.
\end{definition}


We now prove the following lemma regarding the product of a vector with a homogenous multivector.
\begin{lemma}\label{l:v-mv-product}
	The inner and outer products of a vector with a homogeneous multivector have the following expressions:
	\begin{align*}
		& a | A_r = \g{aA_r}{r-1} = \frac{1}{2}(aA_r-(-1)^rA_ra) \\
		& a \wedge A_r = \g{aA_r}{r+1} = \frac{1}{2}(aA_r+(-1)^rA_ra)
	\end{align*}
\end{lemma}

\begin{proof}
	We shall assume that $A_r$ is an $r$-blade: the case of a homogeneous multivector follows directly using distributivity of the geometric product (since an $r$-grade homogeneous multivector is a sum of $r$-blades).

	Recall the definition of the inner product between two vectors (def. \ref{d:inner-product1})
	\[a | b = \frac{1}{2}(ab + ba)\]
	We can reverse it to obtain:
	\[ab = 2 a | b - ba\]
	Repeated application of the above allows us to permute indices in a product, as follows:
	\begin{align*}
		aA_r = aa_1a_2 \ldots a_r = &2a|a_1a_2 \ldots a_r - a_1aa_2 \ldots a_r \\
		= &2a|a_1a_2 \ldots a_r - 2a|a_2a_1 \ldots a_r + a_1aa_2 \ldots a_r \\
		= &\ldots \\
		= &2 \sum_{k=1}^r (-1)^{k+1}a|a_k a_1 \ldots \check{a_k} \ldots a_r + (-1)^ra_1a_2 \ldots a_ra \\
		= &\sum_{k=1}^r (-1)^{k+1}a|a_k a_1 \ldots \check{a_k} \ldots a_r + \sum_{k=1}^r (-1)^{k+1}a|a_k a_1 \ldots \check{a_k} \ldots a_r \\
		&+ (-1)^ra_1a_2 \ldots a_ra
	\end{align*}

	Notice that the first term above has grade $r-1$: we will demonstrate that this is indeed the lowest grade term. For now, we preemptively denote it $a | A_r$.
	
	Let us rewrite the sum using invertibility of vectors (th. \ref{l:invertibility}):
	\begin{align*}
		a | A_r &= \sum_{k=1}^r (-1)^{k+1}a|a_k a_1 \ldots \check{a_k} \ldots a_r\\
				&= \sum_{k=1}^r (-1)^{k+1}a|a_k a_k^{-1}a_k a_1 \ldots \check{a_k} \ldots a_r\\
				&= \sum_{k=1}^r a|a_k a_k^{-1} A_r 
	\end{align*}

	Substracting the above from $aA_r$ and factoring:
	\[aA_r - a | A_r = (a - \sum_{k=1}^r a|a_k a_k^{-1})A_r \equiv bA_r\]
	
	Since by construction, $b | a_k = 0 \forall k \in \{1, \ldots r\}$, it follows by Corollary \ref{c:orthogonality} that the above is a product of $r+1$ anticommuting vectors and thus has grade $r+1$ by Definition \ref{d:k-blades}. We are justified in writing:

	\begin{align*}
		aA_r &= a | A_r + a \wedge A_r \\
		a | A_r &= \sum_{k=1}^r (-1)^{k+1}a|a_k a_1 \ldots \check{a_k} \ldots a_r \\
		a \wedge A_r &= a | A_r + (-1)^rA_ra
	\end{align*}

	From which the lemma follows straightforwardly by substituting the third expression into the first:
	\begin{align*}
		aA_r &= 2a | A_r + (-1)^rA_ra \Rightarrow a | A_r = \frac{1}{2} (aA_r - (-1)^rA_ra) \\
		aA_r &= 2a \wedge A_r - (-1)^rA_ra \Rightarrow a \wedge A_r = \frac{1}{2} (aA_r + (-1)^rA_ra)
	\end{align*}
\end{proof}


The above proof is due to \textit{Hestenes} (p. 8-10)\cite{ga-origin}

Using the above lemma, we can prove the following important property of the geometric product between homogeneous multivectors.
\begin{theorem}[Product of Homogeneous Multivectors]\label{t:homog-product}
	The product of homogeneous multivectors $A_r, B_s$ can be decomposed as follows:
	\[A_rB_s = \sum_{k=0}^{\min\{r,s\}} \g{A_rB_s}{\mg{s-r}+2k}\]
\end{theorem}

\begin{proof}
	We prove this by induction on $r \leq s$ when $A_r$ and $B_s$ are simple $r$- and $s$-vectors respectively.

	The case $r=1,\ s=1$ is true by Definition \ref{t:geometric-product}.
	The case $r=1,\ s>1$ is true by Lemma \ref{l:v-mv-product}.
	Assume the expression holds for $r = q,\ s > r$, we show that it holds for $r+1$:
	\begin{alignat*}{2}
		A_{r+1}B_s &=\ &&a_{r+1}A_rB_s = a_{r+1}\sum_{k=0}^r \g{A_rB_s}{s-r+2k} \\
				   &= &&\sum_{k=0}^r a_{r+1}\g{A_rB_s}{s-r+2k} \\
				   &= &&\sum_{k=0}^r \left[a_{r+1} | \g{A_rB_s}{s-r+2k} + a_{r+1} \wedge \g{A_rB_s}{s-r+2k}\right]\\
				   &=\ &&a_{r+1} | \g{A_rB_s}{s-r} \\
				   &  &&+ \sum_{k=1}^r \left[a_{r+1} | \g{A_rB_s}{s-r+2k} + a_{r+1} \wedge \g{A_rB_s}{s-r+2(k-1)}\right]\\
				   &  &&+ a_{r+1}\wedge \g{A_rB_s}{s+r}
	\end{alignat*}
	where in the last step, we have grouped terms together by grade:
	\begin{alignat*}{2}
		&\g{A_{r+1}B_s}{s-(r+1)} &&\equiv a_{r+1} | \g{A_rB_s}{s-r} \\
		&\g{A_{r+1}B_s}{s-(r+1)+2k} &&\equiv a | \g{A_rB_s}{s-r+2k} + a \wedge \g{A_rB_s}{s-r+2(k-1)} \\
		&\g{A_{r+1}B_s}{r+1+s} &&\equiv a \wedge \g{A_rB_s}{s+r}
	\end{alignat*}

	The case where $s \leq r$ follows by induction on $s$ (the same argument as above).
	The general case for homogeneous multivectors follows by distributivity of the geometric product.
\end{proof}

The above proof is due to \textit{Chisolm} \cite[p. 20-21]{ga-chisolm}


The above proof is due to \textit{Chisolm} (p. 20-21)\cite{ga-chisolm}

As a corollary, we obtain our sought-after result.
\begin{corollary}[Even Subalgebras]\label{c:even-subalgebras}
	The set of even-grade elements of a geometric algebra constitutes a subalgebra.
\end{corollary}



Moreover, we are now equipped to provide general, explicit definitions for the inner and outer products in the case of arbitrary multivectors.
\begin{definition}[Generalized Inner Product]
	On homogeneous multivectors ($s>r$):
	\[A_r | B_s \equiv \g{A_rB_s}{s-r}\]
	%represents the orthogonal complement of the smaller space in the larger space (left\right)
	
	On arbitrary multivectors:
	\[A | B \equiv \sum_{r=0}^{\gr{A}}\sum_{s=r}^{\gr{B}} \g{A}{r} | \g{B}{s}\]
\end{definition}

\begin{definition}[Generalized Outer Product]
	On homogeneous multivectors ($s>r$):
	\[A_r \wedge B_s \equiv \g{A_rB_s}{s+r}\]
	%represents the orthogonal complement of the smaller space in the larger space (left\right)
	
	On arbitrary multivectors:
	\[A \wedge B \equiv \sum_{r=0}^{\gr{A}}\sum_{s=r}^{\gr{B}} \g{A}{r} \wedge \g{B}{s}\]
\end{definition}


