\section{The Universal Geometric Algebra}
\subsection{The UGA as a \textcolor{red}{Category}}

The fundamental concept in Geometric Algebra is that of the \textbf{Universal Geometric Algebra} (UGA for short). It is usually formulated as an infinite-dimensional abstract algebra obeying a certain set of axioms, within which all the Geometric Algebras are contained. \textcolor{red}{CITATION?!}

I have not been able to convince myself that the standard formulation is justified without addressing the peculiarities of infinite-dimensional inner product spaces; so we shall do otherwise, and define it as follows:
\begin{definition}[Universal Geometric Algebra]
	The \textbf{Universal Geometric Algebra} is the \textcolor{red}{category} of algebras obeying a specific set of axioms. Its elements (the Geometric or Clifford Algebras) are specified by the choice of a finite-dimensional inner product space over the reals.
\end{definition}

A \textbf{UGA} is then a template for Geometric Algebras: given a finite-dimensional linear space, and an inner product, there exists a unique axiom-abiding algebra which contains the linear space and whose symmetrized product corresponds to the prescribed inner product. We will later on observe, that this is in fact the Universal Property of Clifford Algebras.

A note on nomenclature is urgently in order: what we call \textbf{Universal Geometric Algebra} is in fact not an algebra; we will continue to refer to it as such, and in fact we will discuss the generally applicable definitions and results in the context of an abstract algebra whose formal product obeys the axioms, and whose abstract linear space will always be assumed to be of a sufficiently high-dimension such that it does not get in the way of the argument.

\subsection{Axioms, Elements and Products}

A Geometric Algebra $\G$  is a finite-dimensional unitary associative algebra over the reals obeying the following axioms:
\begin{axiom}\label{a:ga-axiom1}
    $\G$ contains $\R$ as a subalgebra and a real (finite-dimensional) vector space $\V$ as a subspace; these generate the entire algebra. We call elements of $\R$ scalars, and elements of $\V$ vectors.
\end{axiom}

\begin{axiom}\label{a:ga-axiom2}
	The formal product of a scalar and a vector corresponds with the multiplication by a scalar of the vector space.
\end{axiom}

\begin{axiom}\label{a:ga-axiom3}
    The square of any vector is a real number.
\end{axiom}

\begin{axiom}\label{a:ga-axiom4}
	The formal product on $\V$ is positive-definite, i.e.:
\[\forall v \neq 0 \in \V \quad vv > 0\]
\end{axiom}
\begin{remark}\label{r:ga-axiom4}
	The above axiom distinguishes a Clifford Algebra from a Geometric Algebra in our treatment: a Clifford Algebra only requires that the symmetrized product be a quadratic form. Positive-Definiteness is usually not required in standard treatments of Geometric Algebra and thus the terms are often used interchangeably: we will not be doing so in this thesis.
\end{remark}

\begin{axiom}\label{a:ga-axiom5}
%	The antisymmetrized product of linearly independent vectors	produces an element which does not belong to $\R \bigoplus \V$, we call this product the \textbf{exterior product} and denote it $u \wedge v = \frac{1}{2} (uv - vu)$. 
	The product of linearly independent vectors produces an element that is not in the linear span of its factors: i.e. if $\{v_i\}_{i=1}^n$ is a linearly independent set, then:
	\[\prod_{i=1}^n v_i \notin \mathrm{span}\{\prod_{j=1}^n v_{i_j} \forall i_j : \{1..n\} \to \{0..n\} \} \equiv F(\{v_i\}_{i=1}^n)\]
	were we define $v_0 = 1$ for notational convenience.
\end{axiom}

We refer to the formal product of such an algebra, as a \textbf{geometric product}.

An immediate result of axioms \ref{a:ga-axiom3} and \ref{a:ga-axiom4} is:
\begin{lemma}\label{l:invertibility}
	All vectors are invertible with respect to the geometric product and the inverse is given by:
	\[v^{-1} = \frac{v}{\mg{v}^2}\]
\end{lemma}



We first prove the following lemma.
\begin{lemma}[Inner Product]\label{l:inner-product}
	The symmetrized product of two vectors, $u|v = \frac{1}{2}(uv + vu)$, is an inner product.
\end{lemma}

\begin{proof}
	Consider the following expression, and recall Axiom \ref{a:ga-axiom3}:
    \begin{align*}
        &(u+v)^2 = u^2 + v^2 + uv + vu \Leftrightarrow \\
        &2u|v = (u+v)^2 - u^2 - v^2 \in \R
    \end{align*}
	So the symmetrized product maps into $\R$ (i.e. $| : \V \times \V \to \R$). By Definition \ref{d:algebra}, the geometric product is bilinear and so is the symmetrized product as it is a linear function of the bilinear products; moreover, by Axiom \ref{a:ga-axiom4}, it is also positive-definite. We conclude that $u|v = uv + vu$ is indeed an inner product on $\V$.
	
\end{proof}


From the above, we obtain the principal property of the geometric product.
\begin{theorem}[Geometric Product of Vectors]\label{t:geometric-product}
	The geometric product of two vectors can be broken down into an inner and an exterior product: \[uv = u | v + u \wedge v\]
\end{theorem}

\begin{proof}
	We first show how the geometric product decomposes into a symmetric and antisymmetric part:
	\[uv = \frac{1}{2}(uv + uv) = \frac{1}{2}(uv + vu + uv - vu) = \frac{1}{2}(uv + vu) + \frac{1}{2}(uv - vu)\]
	The result follows immediately from Lemma \ref{l:inner-product} and the above remarks.
\end{proof}


As a corollary, we obtain the following self-evident propositions.
\begin{corollary}
	The exterior product of a linearly dependent set of vectors is zero.
\end{corollary}

\begin{corollary}\label{c:orthogonality}
	A pair of vectors are orthogonal if and only if they anticommute with respect to the geometric product.
\end{corollary}



We will now go on to consider general expressions and properties of products between arbitrary multivectors; but first, we ought to familiarise ourselves with some specific terminology.
\begin{definition}
	We refer to reals $a \in \R \subset \G(\V)$ as \textbf{scalars}, or $\textit{grade-0}$ vectors.
\end{definition}
\begin{definition}
	Elements $v \in \V \subset \G(\V)$ are called \textbf{$1$-vectors}, or simply vectors.
\end{definition}
\begin{definition} \label{d:k-blades}
	A \textbf{$k$-blade} is a product $e_{i_1}e_{i_2} \ldots e_{i_k}$ of orthogonal vectors; these are also called simple $k$-vectors.
\end{definition}
\begin{definition}
	A \textbf{versor} is an arbitrary product $v_1v_2 \ldots v_k$ of vectors.
\end{definition}
\begin{definition}
	\textbf{Multivectors} are finite sums of versors. By Axiom \ref{a:generation}, every element $V \in \G(\V)$ is a multivector.
\end{definition}

A brief note on conventions: here on out, whenever left unspecified, the precedence of products is inner$\to$wedge$\to$geometric.

First off, we generalize the definition of the inner and outer products to homogeneous multivectors.
\begin{definition}\label{d:inner-product1}
	The inner product $A_r|B_s$ between two homogeneous multivectors $A_r$ and $B_s$ is the lowest-possible grade term in their product. We will later see that this is the $\mg{r-s}$-grade term.
\end{definition}

\begin{definition}\label{d:outer-product1}
	The outer product $A_r \wedge B_s$ between two homogeneous multivectors $A_r$ and $B_s$ is the highest-possible grade term in their product. We will later see that this is the $\mg{r+s}$-grade term; if the product does not have such a term, then we say that the outer product is 0.
\end{definition}


We now prove the following steppingstone lemma regarding the product of a vector with a homogenous multivector.
\begin{lemma}\label{l:v-mv-product}
	The inner and outer products of a vector with a homogeneous multivector have the following expressions:
	\begin{align*}
		& a | A_r = \g{aA_r}{r-1} = \frac{1}{2}(aA_r-(-1)^rA_ra) \\
		& a \wedge A_r = \g{aA_r}{r+1} = \frac{1}{2}(aA_r+(-1)^rA_ra)
	\end{align*}
\end{lemma}

\begin{proof}
	We shall assume that $A_r$ is an $r$-blade: the case of a homogeneous multivector follows directly using distributivity of the geometric product (since a grade-$r$ homogeneous multivector is a sum of $r$-blades).

	Recall Lemma \ref{l:inner-product} 
	\[a | b = \frac{1}{2}(ab + ba)\]
	We can reverse it to obtain:
	\[ab = 2 a | b - ba\]
	Repeated application of the above allows us to permute indices in a product, as follows:
	\begin{align*}
		aA_r = aa_1a_2 \ldots a_r = &2a|a_1a_2 \ldots a_r - a_1aa_2 \ldots a_r \\
		= &2a|a_1a_2 \ldots a_r - 2a|a_2a_1 \ldots a_r + a_1a_2aa_3 \ldots a_r \\
		= &\ldots \\
		= &2 \sum_{k=1}^r (-1)^{k+1}a|a_k a_1 \ldots \check{a_k} \ldots a_r + (-1)^ra_1a_2 \ldots a_ra \\
		= &\sum_{k=1}^r (-1)^{k+1}a|a_k a_1 \ldots \check{a_k} \ldots a_r + \sum_{k=1}^r (-1)^{k+1}a|a_k a_1 \ldots \check{a_k} \ldots a_r \\
		&+ (-1)^ra_1a_2 \ldots a_ra
	\end{align*}

	Notice that the first term above (denote it $T_1$ for convenience) has grade $r-1$: we will prove that this is indeed the lowest grade term by showing that $aA_r - T_1$ has grade $r+1$.
	
	Let us rewrite the sum using invertibility of vectors (Lemma \ref{l:invertibility}):
	\begin{align*}
		T_1 &= \sum_{k=1}^r (-1)^{k+1}a|a_k a_1 \ldots \check{a_k} \ldots a_r\\
				&= \sum_{k=1}^r (-1)^{k+1}a|a_k a_k^{-1}a_k a_1 \ldots \check{a_k} \ldots a_r\\
				&= \sum_{k=1}^r a|a_k a_k^{-1} A_r 
	\end{align*}

	Substracting the above from $aA_r$ and factoring:
	\[aA_r - a | A_r = (a - \sum_{k=1}^r a|a_k a_k^{-1})A_r \equiv bA_r\]
	
	Since $a_k^{-1} = \mg{a_k}^{-2}a_k$ (Lemma \ref{l:invertibility}), we have by construction that
	\[b | a_k = (a - \sum_{k=1}^r (a|a_k) \mg{a_k}^{-2}a_k)|a_k = a|a_k - a|a_k  = 0 \]

	It follows by Corollary \ref{c:orthogonality} that $bA_r$ is a product of $r+1$ orthogonal vectors and thus has grade $r+1$ by Definition \ref{d:grade}. We are justified in writing:

	\begin{align*}
		aA_r &= a | A_r + a \wedge A_r \\
		a | A_r &= \sum_{k=1}^r (-1)^{k+1}a|a_k a_1 \ldots \check{a_k} \ldots a_r \\
		a \wedge A_r &= a | A_r + (-1)^rA_ra
	\end{align*}

	From which the lemma follows straightforwardly by substituting the third expression into the first:
	\begin{align*}
		aA_r &= 2a | A_r + (-1)^rA_ra \Rightarrow a | A_r = \frac{1}{2} (aA_r - (-1)^rA_ra) \\
		aA_r &= 2a \wedge A_r - (-1)^rA_ra \Rightarrow a \wedge A_r = \frac{1}{2} (aA_r + (-1)^rA_ra)
	\end{align*}
\end{proof}

The above proof is due to \textit{Hestenes} \cite[p. 8-10]{ga-origin}


Using the above lemma, we can prove the following important property of the geometric product between homogeneous multivectors.
\begin{theorem}[Product of Homogeneous Multivectors]\label{t:homog-product}
	The product of homogeneous multivectors $A_r, B_s$ (for $s \leq r$) can be decomposed as follows:
	\[A_rB_s = \sum_{k=0}^r \g{A_rB_s}{s-r+2k}\]
\end{theorem}

%\begin{proof}
	We prove this by induction on $r \leq s$ when $A_r$ and $B_s$ are simple $r$- and $s$-vectors respectively.

	The case $r=1,\quad s=1$ is true by Definition \ref{t:geometric-product}.
	The case $r=1,\quad s>1$ is true by Lemma \ref{l:v-mv-product}.
	Assume the expression holds for $r = q,\quad s > r$, we show that it holds for $r+1$:
	\begin{align*}
		A_{r+1}B_s = &a_{r+1}A_rB_s = a_{r+1}\sum_{k=0}^r \g{A_rB_s}{s-r+2k} = \\
		= &\sum_{k=0}^r a_{r+1}\g{A_rB_s}{s-r+2k} = \\
		= &\sum_{k=0}^r \left[a_{r+1} | \g{A_rB_s}{s-r+2k} + a_{r+1} \wedge \g{A_rB_s}{s-r+2k}\right]\\
		= &a | \g{A_rB_s}{s-r} \\
		  + &\sum_{k=1}^r \left[a_{r+1} | \g{A_rB_s}{s-r+2k} + a_{r+1} \wedge \g{A_rB_s}{s-r+2(k-1)}\right]\\
		  + &a\wedge \g{A_rB_s}{s+r}
	\end{align*}
	where in the last step, we have grouped terms together by grade:
	\begin{align*}
		&\g{A_{r+1}B_s}{s-(r+1)} \equiv a_{r+1} | \g{A_rB_s}{s-r} \\
		&\g{A_{r+1}B_s}{s-(r+1)+2k} \equiv a | \g{A_rB_s}{s-r+2k} + a \wedge \g{A_rB_s}{s-r+2(k-1)} \\
		&\g{A_{r+1}B_s}{r+1+s} \equiv a \wedge \g{A_rB_s}{s+r}
	\end{align*}

	The case where $s \leq r$ follows by induction on $s$ (the same argument as above).
	The general case for homogeneous multivectors follows by distributivity of the geometric product.
\end{proof}

\textcolor{red}{ATTRIBUTE APPROACH OF THE PROOF TO HESTENES}

As a corollary, we obtain our sought-after result.
\begin{corollary}[Even Subalgebras]\label{c:even-subalgebras}
	The set of even-grade elements of a geometric algebra constitutes a subalgebra.
\end{corollary}



Moreover, we are now equipped to provide general, explicit definitions for the inner and outer products in the case of arbitrary multivectors.
\input{src/definitions/g-inner-product2}
\textcolor{red}{THE INNER PRODUCT IS A GRADE LOWERING OPERATION, REPRESENTING WHAT? represents the orthogonal complement of the smaller space in the larger space (left-right)}
\begin{corollary}
	(Generalized Outer Product)

	On homogeneous multivectors:
	\[A_r \wedge B_s \equiv \g{A_rB_s}{s+r}\]
	%represents the orthogonal complement of the smaller space in the larger space (left\right)
	
	On arbitrary multivectors:
	\[A \wedge B \equiv \sum_{r=0}^{\gr{A}}\sum_{s=r}^{\gr{B}} \g{A}{r} \wedge \g{B}{s}\]
\end{corollary}

\textcolor{red}{THE EXTERIOR PRODUCT IS A GRADE LOWERING OPERATION, REPRESENTING WHAT? represents the orthogonal complement of the smaller space in the larger space (left-right)}

\newpage

\section{Constructing a Geometric Algebra}

\textcolor{red}{DISCUSS THE VALIDITY OF THE AXIOMS AND PROVE THAT GEOMETRIC ALGEBRAS ARE SIMPLY CLIFFORD ALGEBRAS (DECIDE IF I SHOULD RELAX POSITIVE-DEFINITENESS REQUIREMENT AND DRAW DISTINCTION THERE OR IF I DRAW DISTINCTION AT KEEPING THE INNER PRODUCT FORMAL OR BOTH); PROVE UNIVERSAL PROPERTY}

\textcolor{red}{OR \\ PROVE UNIVERSAL PROPERTY (FORMAL CONSTRUCTION) FROM AXIOM, DISCUSS BASES (PRACTICAL CONSTRUCTION; before universal property?), GRADE-DIMENSION AND PSEUDOSCALAR}

