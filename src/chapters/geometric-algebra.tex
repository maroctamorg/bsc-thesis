\section{The Universal Geometric Algebra}
\subsection{The UGA as a \textcolor{red}{Category}}

The fundamental concept in Geometric Algebra is that of the \textbf{Universal Geometric Algebra} (UGA for short). It is usually formulated as an infinite-dimensional abstract algebra obeying a certain set of axioms, within which all the Geometric Algebras are contained. \textcolor{red}{CITATION?!}

I have not been able to convince myself that the standard formulation is justified without addressing the peculiarities of infinite-dimensional inner product spaces; so we shall do otherwise, and define it as follows:
\begin{definition}[Universal Geometric Algebra]
	The \textbf{Universal Geometric Algebra} is the \textcolor{red}{category} of algebras obeying a specific set of axioms. Its elements (the Geometric or Clifford Algebras) are specified by the choice of a finite-dimensional inner product space over the reals.
\end{definition}

A \textbf{UGA} is then a template for Geometric Algebras: given a finite-dimensional linear space, and an inner product, there exists a unique axiom-abiding algebra which contains the linear space and whose symmetrized product corresponds to the prescribed inner product. We will later on observe, that this is in fact the Universal Property of Clifford Algebras.

A note on nomenclature is urgently in order: what we call \textbf{Universal Geometric Algebra} is in fact not an algebra; we will continue to refer to it as such, and in fact we will discuss the generally applicable definitions and results in the context of an abstract algebra whose formal product obeys the axioms, and whose abstract linear space will always be assumed to be of a sufficiently high-dimension such that it does not get in the way of the argument.

\subsection{The Axioms}

A Geometric Algebra $\G$  is a finite-dimensional unitary associative algebra over the reals obeying the following axioms:
\begin{axiom}\label{a:ga-axiom1}
    $\G$ contains $\R$ as a subalgebra and a real (finite-dimensional) vector space $\V$ as a subspace; these generate the entire algebra. We call elements of $\R$ scalars, and elements of $\V$ vectors.
\end{axiom}

\begin{axiom}\label{a:ga-axiom2}
	The formal product of a scalar and a vector corresponds with the multiplication by a scalar of the vector space.
\end{axiom}

\begin{axiom}\label{a:ga-axiom3}
    The square of any vector is a real number.
\end{axiom}

\begin{axiom}\label{a:ga-axiom4}
	The formal product on $\V$ is positive-definite, i.e.:
\[\forall v \neq 0 \in \V \quad vv > 0\]
\end{axiom}
\begin{remark}\label{r:ga-axiom4}
	The above axiom distinguishes a Clifford Algebra from a Geometric Algebra in our treatment: a Clifford Algebra only requires that the symmetrized product be a quadratic form. Positive-Definiteness is usually not required in standard treatments of Geometric Algebra and thus the terms are often used interchangeably: we will not be doing so in this thesis.
\end{remark}

\begin{axiom}\label{a:ga-axiom5}
%	The antisymmetrized product of linearly independent vectors	produces an element which does not belong to $\R \bigoplus \V$, we call this product the \textbf{exterior product} and denote it $u \wedge v = \frac{1}{2} (uv - vu)$. 
	The product of linearly independent vectors produces an element that is not in the linear span of its factors: i.e. if $\{v_i\}_{i=1}^n$ is a linearly independent set, then:
	\[\prod_{i=1}^n v_i \notin \mathrm{span}\{\prod_{j=1}^n v_{i_j} \forall i_j : \{1..n\} \to \{0..n\} \} \equiv F(\{v_i\}_{i=1}^n)\]
	were we define $v_0 = 1$ for notational convenience.
\end{axiom}


\subsection{The Elements}

\subsection{The Products}

\section{Clifford Algebras}

\section{The Euclidean Geometric Algebras: $\G(\E_1), \G(\E_2), \G(\E_3)$ }

\subsection{$\G(\E_1)$}

\subsection{$\G(\E_2)$}

\subsection{$\G(\E_3)$}

