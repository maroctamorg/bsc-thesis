\section{The Universal Geometric Algebra}
\subsection{The UGA as an Abstract Algebra Template\textcolor{red}{[CATEGORY]}}

The fundamental concept in Geometric Algebra is that of the \textbf{Universal Geometric Algebra} (UGA for short). It is usually formulated as an infinite-dimensional abstract algebra obeying a certain set of axioms, within which all the Geometric Algebras are contained. \textcolor{red}{CITATION?!}

We shall do otherwise, and define it as follows:
\begin{definition}[Universal Geometric Algebra]
	The \textbf{Universal Geometric Algebra} is the \textcolor{red}{category} of algebras obeying a specific set of axioms. Its elements (the Geometric or Clifford Algebras) are specified by the choice of a finite-dimensional inner product space over the reals.
\end{definition}

A \textbf{UGA} is then a template for Geometric Algebras: given a finite-dimensional linear space, and an inner product, there exists a unique axiom-abiding algebra which contains the linear space and whose symmetrized product corresponds to the prescribed inner product.

\subsection{The Axioms}

\subsection{The Universal Property}

\subsection{The Elements}

\subsection{The Products}

\section{Clifford Algebras}

\section{The Euclidean Geometric Algebras: $\G(\E_1), \G(\E_2), \G(\E_3)$ }

\subsection{$\G(\E_1)$}

\subsection{$\G(\E_2)$}

\subsection{$\G(\E_3)$}

