%\section{The Geometric Family of Algebras}\label{s:uga}
%
%In the standard formalism, the fundamental concept in Geometric Algebra is that of the \textbf{Universal Geometric Algebra} (UGA for short): an infinite-dimensional abstract algebra obeying a certain set of axioms, within which all the Geometric Algebras are contained (as subalgebras) \cite{ga-origin}. 
%
%For our purposes, whereby we will limit ourselves to geometric algebras over finite-dimensional inner product spaces, I have found this to be unnecessary and not the best suited approach. Instead we shall formulate as our starting point, a family of geometric algebras:
%\begin{definition}[Geometric Family of Algebras]
%	The \textbf{Geometric Family of Algebras} is a family of algebras obeying a specific set of axioms. Its elements (the Geometric Algebras) are specified by the choice of a finite-dimensional inner product space over the real numbers.
%\end{definition}
%
%A \textbf{GFA} is then a template for Geometric Algebras: given a finite-dimensional linear space, and an inner product, there exists a unique axiom-abiding algebra which contains the linear space and whose symmetrized product corresponds to the prescribed inner product.
%
%We will, however, discuss the generally applicable definitions and results in the context of an abstract algebra whose formal product obeys the axioms, and whose abstract linear space will always be assumed to be of a sufficiently high (finite) dimension such that it does not get in the way of the argument.
%
%\newpage

\section{Axioms and Definitions}\label{s:axioms-definitions}

%The Geometric Algebra $\G$ of a finite-dimensional inner product space $\V$ is the unital associative algebra over the reals which obeys the following axioms:
%\begin{axiom}\label{a:ga-axiom1}
    $\G$ contains $\R$ as a subalgebra and $\V$ as a subspace; these generate the entire algebra. We call elements of $\R$ scalars, and elements of $\V$ vectors.
\end{axiom}

\begin{axiom}\label{a:ga-axiom2}
	The formal product of a scalar and a vector corresponds with the multiplication by a scalar of the vector space.
\end{axiom}

\begin{axiom}\label{a:ga-axiom3}
    The square of a vector corresponds to its inner product with itself.
	\[\forall v \neq 0 \in \V \quad v^2 \equiv vv = v|v = |v|^2\]
\end{axiom}

%\begin{axiom}\label{a:ga-axiom4}
%	The formal product on $\V$ is positive-definite, i.e.:
%\[\forall v \neq 0 \in \V \quad vv > 0\]
%\end{axiom}

\begin{remark}\label{r:ga-axiom3}
	The above axiom distinguishes a Clifford Algebra from a Geometric Algebra in our treatment: a Clifford Algebra only requires a quadratic form on a vector space $\V$. Standard treatments of Geometric Algebra also do not make such a strict requirement and thus the terms are often used interchangeably: we will not be doing so in this thesis.
\end{remark}

%\begin{axiom}\label{a:ga-axiom5}
%%	The antisymmetrized product of linearly independent vectors	produces an element which does not belong to $\R \bigoplus \V$, we call this product the \textbf{exterior product} and denote it $u \wedge v = \frac{1}{2} (uv - vu)$. 
%	The product of linearly independent vectors produces an element that is not in the linear span of its factors: i.e. if $\{v_i\}_{i=1}^n$ is a linearly independent set, then:
%	\[\prod_{i=1}^n v_i \notin \mathrm{span}\{\prod_{j=1}^n v_{i_j} \forall i_j : \{1..n\} \to \{0..n\} \} \equiv F(\{v_i\}_{i=1}^n)\]
%	were we define $v_0 = 1$ for notational convenience.
%\end{axiom}

%We refer to the product of such an algebra as a \textbf{geometric product}.

%\begin{remark}
%	That the algebra is the 'freest' simply means that we specify the product just enough that it satisfies the axioms, and impose no further relations. One could for example define a geometric product on $\R^3$ as the sum of the inner and cross products: the generated algebra would satisfy the axioms, however it would not be the freest since the cross product imposes the additional equivalence relation:
%	\[e_ie_{[i+1]_3} \equiv e_{[i+2]_3} \quad \forall i \in \{1,2,3\}\]
%	where $[a]_3$ is a shorthand notation for $a$ mod $3$.
%	A modern, abstract formulation of the concept has this property naturally since the Geometric Algebra is constructed from more abstract objects by imposing only the required relations: I have not taken this route given the scope and target audience of the thesis. %\textcolor{red}{FETCH A REFERENCE?!}
%\end{remark}

Let $(\V, \cdot|\cdot )$ be a real finite-dimensional inner product space, $\G$ be a unital associative algebra over $\R$ and $l: \V \to \G$ be a linear map.

\begin{definition}[Geometric Algebra]
	The pair $(\G, l)$ is the \textbf{Geometric Algebra} for the space $(\V, \cdot|\cdot)$ when it satisfies the following axioms:
\end{definition}
\begin{axiom}\label{a:generation}
	$\G$ is spanned by the products of elements in the set
	\[\{l(v)|v \in \V\} \bigcup \{1_{\G}\}\]
	We say that the algebra is generated by the above set. 
\end{axiom}
\begin{axiom}\label{a:square}
	For all vectors $v \in \V$, the square of their mapping $l(v)$ corresponds to the square of their magnitude, i.e.:
	\[\forall v \in \V: l(v)^2 = v|v1_{\G} = |v|^21_{\G}\]
\end{axiom}
\begin{axiom}\label{a:universal}
	Given any unital associative algebra $A$ over $\R$ and any linear map $j: \V \to A$ such that
	\[\forall v \in \V: j(v)^2 = (v|v)1_A\]
	there is a unique algebra homomorphism $f: \G \to A$ such $f \odot i = j$.

	\begin{center}
		\large
		\begin{tikzcd}[row sep=large,column sep=large]
			\V \arrow[r, "i"] \arrow[rd, "j" ]
			& \G \arrow[d, "f", dashrightarrow] \\
			& A
		\end{tikzcd}
	\end{center}
\end{axiom}


Before anything else, we present a few theorems and remarks regarding existence and universality.
\begin{theorem}[Uniqueness]\label{t:uniqueness}
	The Geometric Algebra $(\G,l)$ of an inner product space $(\V, \cdot|\cdot)$, if it exists, is unique up to isomorphism.
\end{theorem}

\begin{proof}
	If $(A, i)$ and $(B, j)$ are two Geometric Algebras of $(\V, \cdot|\cdot)$, then by Axiom \ref{a:universal} there exist algebra homomorphisms:
	\begin{align*}
		&f: A \to B \text{ s. that } f \odot i = j \\
		&g: B \to A \text{ s. that } g \odot j = i
	\end{align*}

	It follows that the compositions obey:
	\begin{alignat*}{2}
		&g \odot f&&: A \to A \text{ s. that } i = (g \odot f) \odot i \\
		& &&\Rightarrow g \odot f = id_A : A \to A \\
		&f \odot g&&: B \to B \text{ s. that } j = (f \odot g) \odot j \\
		& &&\Rightarrow f \odot g = id_B : B \to B
	\end{alignat*}

	We conclude that $g = f^{-1}$ and thus the homomorphisms are 1-to-1 so that the Geometric Algebras are isomorphic.
\end{proof}


\begin{definition}[Existence and Dimensionality]\label{t:existence-dimension}
	For every finite-dimensional real inner product space $(\V, \cdot|\cdot)$ there exists a Geometric Algebra $\G$ with dimension $2^n$, where $n = \dim(\V)$.
\end{definition}

A generalized proof of existence and dimensionality is outside the scope of this thesis; later on, we will construct the Geometric Algebras of 1, 2 and 3-dimensional Euclidean Spaces explicitly and that will be sufficient for our discussion. For the interested reader, you may refer to a general proof by construction in \cite[Section 3.2]{clifford-algebra}

\begin{remark}
	From now on, we will make the identifications
	\begin{align*}
		&l(v) \equiv v \ \forall v \in \V \\
		&\alpha 1_A \equiv \alpha \ \forall \alpha \in \R
	\end{align*}
	We will also omit explicit mention of the mapping $l: \V \to \G$, and instead refer to a geometric algebra simply as $\G$. This is justified due to the linearity of the mapping $l: \V \to A$, and the bilinearity of the product of an algebra, which implies that $\alpha 1_A \cdot \beta 1_A  + \gamma 1_A = (\alpha \beta + \gamma) 1_A$.
\end{remark}

We are now equipped to prove some basic properties about geometric algebras.

Two results follow immediately from Axiom \ref{a:square}:
\begin{lemma}\label{l:invertibility}
	All vectors are invertible with respect to the geometric product and the inverse is given by:
	\[v^{-1} = \frac{v}{\mg{v}^2}\]
\end{lemma}



\begin{lemma}[Inner Product]\label{l:inner-product}
	The symmetrized product of two vectors, \[s(u,v) = \frac{1}{2}(uv + vu)\] corresponds to the inner product on $\V$, i.e.: \[s(u,v) = u|v \forall u,v \in \V\]
\end{lemma}

\begin{proof}
	Consider the following expression, and recall Axiom \ref{a:ga-axiom3}:
    \begin{align*}
        &(u+v)^2 = u^2 + v^2 + uv + vu \Leftrightarrow \\
        &2u|v = (u+v)^2 - u^2 - v^2 \in \R
    \end{align*}
	So the symmetrized product maps into $\R$ (i.e. $| : \V \times \V \to \R$). By Definition \ref{d:algebra}, the geometric product is bilinear and so is the symmetrized product as it is a linear function of the bilinear products; moreover, by Axiom \ref{a:ga-axiom4}, it is also positive-definite. We conclude that $u|v = uv + vu$ is indeed an inner product on $\V$.
	
\end{proof}


\begin{remark}
	We have seen that $\R \bigcup \V \subset \G$ is closed under the symmetric part of the product; the algebra must thus be generated by the antisymmetric part of the product: we denote this bilinear antisymmetric product $u \wedge v$ and call it the \textbf{exterior product}.
\end{remark}
%\begin{definition}[Exterior Product]\label{d:exterior-product}
	The linear rejection $R(\prod_{i=1}^n, F(\{v_i\}_{i=1}^n))$ of a product with regards to the linear span of its factors is denoted
	\[ \bigwedge_{i=1}^n v_i \equiv v_1 \wedge ... \wedge v_n \]
	We call this exterior (or outer) product.
\end{definition}

%We have observed that the axioms place a restriction on the symmetric par of the product between two vectors. The antisymmetric part, on the other hand, we denote $u \wedge v$ and call it the exterior product. This product is specified by Axiom \ref{a:freest}, as presented in the following lemma:
%\input{src/lemmas/outer-product}
%\input{src/proofs/outer-product}
%bears no restriction: since the Geometric Algebra is the most general algebra obeying the axioms, we identify this part of the product with a formal bilinear and antisymmetric product, we call it the \textbf{exterior} product.

One of the main results of this section is the principal property of the geometric product.
\begin{theorem}[Geometric Product of Vectors]
	The geometric product of two vectors can be broken down into an inner and an exterior product: \[uv = u | v + u \wedge v\]
\end{theorem}

\begin{proof}
	We first show how the geometric product decomposes into a symmetric and antisymmetric part:
	\[uv = \frac{1}{2}(uv + uv) = \frac{1}{2}(uv + vu + uv - vu) = \frac{1}{2}(uv + vu) + \frac{1}{2}(uv - vu)\]
	The result follows immediately from Lemma \ref{l:inner-product} and Axiom \ref{a:ga-axiom5}:
	\[uv = u | v + u \wedge v\]
\end{proof}


As a corollary, we obtain the following self-evident propositions.
%\begin{corollary}
	The exterior product of a linearly dependent set of vectors is zero.
\end{corollary}

%\begin{proof}
	PROBLEM: exterior product is currently defined in terms of geometric product only for two vectors...
	Let $\{v_i\}_{i=1}^n$ be a linearly dependent set of vectors, i.e.:
	\[\exists \{\alpha_i\}_{i=1}^{n-1} : v_n \equiv \sum_{i=1}^{n-1} \alpha_i v_i\]
	Then, consider their outer product and recall bilinearity and antisymmetry:
	\begin{align*}
		\bigwedge_{i=1}^n v_i &= v_1 \wedge ... \wedge \sum{i=1}^{n-1} \alpha_i v_i \\
		&= (\bigwedge_{i=1}^{n-1} v_i) \
\end{proof}

\begin{corollary}
	A pair of vectors are orthogonal if and only if they anticommute with respect to the geometric product.
\end{corollary}


\begin{corollary}\label{c:orthonormal-bases}
	An orthonormal basis $\{e_i\}_{i=1}^n$ obeys the following relations:
	\begin{align*}
		e_ie_j &= -e_je_i \\
		e_ie_i &= 1
	\end{align*}
\end{corollary}


Before we move on to consider more general results, we ought to be familiar with some specific terminology.
\begin{definition}
	We refer to reals $a \in \R \subset \G(\V)$ as \textbf{scalars}, or $\textit{grade-0}$ vectors.
\end{definition}
\begin{definition}
	Elements $v \in \V \subset \G(\V)$ are called \textbf{$1$-vectors}, or simply vectors.
\end{definition}
\begin{definition} \label{d:k-blades}
	A \textbf{$k$-blade} is a product $e_{i_1}e_{i_2} \ldots e_{i_k}$ of orthogonal vectors; these are also called simple $k$-vectors.
\end{definition}
\begin{definition}
	A \textbf{versor} is an arbitrary product $v_1v_2 \ldots v_k$ of vectors.
\end{definition}
\begin{definition}
	\textbf{Multivectors} are finite sums of versors. By Axiom \ref{a:generation}, every element $V \in \G(\V)$ is a multivector.
\end{definition}


%A critical lemma to which we will often make implicit reference is the following:
%\begin{lemma}\label{l:ga-expansion}
	Every element $V \in \G(\V)$ has a unique decomposition (with respect to a given basis) as a sum of k-blades.
\end{lemma}

%\begin{proof}
	Every multivector can be written as a finite sum of versors, so it suffices to prove the statement for an arbitrary versor.
	Consider an arbitrary set of vectors expressed in a given orthonormal basis $\{v_i \equiv \sum_{j=1}^n v_i^je_j\}_{i=1}^m$; we expand their product using bilinearity:
	\[\prod_{i=1}^m v_i = \prod_{i=1}^m \sum_{j=1}^n v_i^j e_j = \sum_{j_i : \{1..m\} \to \{1..n\}} \left(\prod_{i=1}^m v_i^{j_i}\right) \left(\prod_{i=1}^m e_{j_i}\right)\]
	It follows from Corollary \ref{c:orthonormal-bases} that the product of the basis vectors in each term is a $k$-blade, where $k$ is the number of unit vectors appearing an odd number of times in the product.
	Uniqueness follows from the freeness of the algebra: the product of orthogonal vectors is a formal product subject only to bilinearity and antisymmetry, so that distinct products of orthonormal basis vectors are linearly independent and form in fact a vector space-basis for the whole algebra.
\end{proof}


\begin{definition}[Basis]\label{d:basis}
	The \textbf{basis} of a Geometric Algebra $\G(\V)$ is simply the basis of the generating inner-product space $\V$.
\end{definition}

\begin{definition}
	A \textbf{frame} of a Geometric Algebra $\G(\V)$ is the set of distinct (up to permutation) $k$-blades formed from an orthogonal basis of $\G(\V)$.
\end{definition}

\begin{remark}
	It is important to note that it is a non-trivial fact of combinatorics and group theory that the sign of a permutation is well-defined. We will make implicit use of this in arguments and definitions that refer to permutation of orthogonal vectors in a product: in practice this means that we are justified in assigning a unique sign to a permutation based on counting the number of pairwise transposition of factors, and saying things such as 'distinct up to permutation'.
\end{remark}
\begin{theorem}[A Frame is a Linear Basis]\label{t:frame}
	Any frame is a linear basis for the geometric algebra.
\end{theorem}

\begin{proof}
	By definition a frame is a linearly independent set, we need only show that a frame spans the algebra. Consider an arbitrary set of vectors expressed in a given orthogonal basis $\{v_i \equiv \sum_{j=1}^n v_i^je_j\}_{i=1}^m$; we expand their product using bilinearity:
	\[\prod_{i=1}^m v_i = \prod_{i=1}^m \sum_{j=1}^n v_i^j e_j = \sum_{j_i : \{1..m\} \to \{1..n\}} \left(\prod_{i=1}^m v_i^{j_i}\right) \left(\prod_{i=1}^m e_{j_i}\right)\]
	It follows from Corollary \ref{c:orthonormal-bases} that the product of the basis vectors in each term is a $k$-blade, where $k$ is the number of unit vectors appearing an odd number of times in the product. The above shows that the product of vectors is indeed a linear combination of elements in the frame generated by the orthogonal basis above.

	A frame can be mapped into the power set of a basis as it constitutes a subset of all the possible combinations of the basis elements without repetition and up to permutation (we identify the unit scalar in the frame with the empty subset of the basis). As such, the maximal number of distinct elements of a frame is $2^n$.

	Since the algebra has dimension $2^n$, the frame must have $2^n$ linearly independent elements and indeed constitute a linear basis for the algebra.
\end{proof}


We are now justified in presenting the following definitions:
\begin{definition}[Grade]\label{d:grade}
	We introduce the grade operator $\g{A}{k}$ which returns the $k$-grade component of $A$.
	The grade of a multivector $\gr{A} \in \N$ is the grade of the maximum-grade term in $A$.
	We say $A$ is a \textbf{homogeneous} multivector of grade $k$ iff it is a sum of $k$-blades for a given $k \in \N$ (when we wish to make it explicit, we denote it $A_k$); otherwise we say $A$ is of mixed grade.
\end{definition}

%\begin{lemma}\label{l:grade-k-2}
	If $\{w_i\}_{i=1}^k$ is an orthogonal set of vectors, then the $k-2$-component of its product is 0 in any orthonormal basis.
\end{lemma}

%\begin{proof}
	Let $\{e_j\}_{j=1}^n$ be an orthonormal basis, and
	\[w_i = \sum_{j=1}^n A_{ij} e_j\]

	The orthogonality of $\{w_i\}_{i=1}^k$ implies that for $a \neq b$:
	\begin{align*}
		w_a|w_b &= (\sum_{j=1}^n A_{aj} e_j)|(\sum_{r=1}^n A_{br} e_r) \\
				&= \sum_{j=1}^n A_{aj}A_{bj} = 0
	\end{align*}

	Now expanding the full product, we have:
	\[w_1...w_k = (\sum_{j_1=1}^n A_{1{j_1}})...(\sum_{j_k=1}^n A_{k{j_k}})\]

	We can index the terms in the above product as follows: at each of the $k$ factors choose one of the terms in the sum (i.e. $A_{it} e_t$), construct a sequence $t(i)$ out of subsequent choices.

	The grade $k-2$ terms (w.r.t. the orthonormal basis) are those terms for which only one choice is repeated; say that the repetition occurs at factors $q$ and $m$ with value $c = t(q) = t(m)$. We can also see this in the other direction, that is to say: given a choice of $q$ and $m$ where the repetition ought to occur, we are free to choose any $c \in \{1..n\}: t(q) = t(m) = c$.

	Thus, we may collect all terms of grade $k-2$ in the following sum:
	\begin{align*}
		\sum_{(q,m) \in \{1..k\}^2} \sum_{c=1}^n (A_{1t(1)}...A_{qc}A_{mc}...A_{kt(k)})e_{t(1)}...\cap{e_{t(q)}}\cap{e_{t(m)}}
	\end{align*}
\end{proof}

\begin{lemma}\label{l:grade-definiteness}
	The grade of a homogeneous multivector is a well-defined property, that is to say every blade has a unique grade that is independent of the choice of orthogonal vectors.
\end{lemma}

\begin{proof}
	First, we prove that for a given orthogonal basis, the grade of a blade is unique, then we show that the grade of a blade is independent of the choice of basis. The result extends directly to general multivectors.

	By Theorem \ref{t:frame}, a blade is simply an element of some frame of $\G$: since a frame is a linear basis for the Geometric Algebra, it follows that frame elements of different grade are linearly independent. The grade is thus unique w.r.t. to a specific choice of frame.


	Let $\{e_j\}_{j=1}^n$ and \[w_i = \sum_{j=1}^n A_{ij} e_j\] be two orthonormal bases (orthogonality suffices, we choose this purely for notational convenience).
	

	The orthogonality of $\{w_i\}_{i=1}^k$ implies that for $a \neq b$:
		
	\[w_a|w_b = (\sum_{j=1}^n A_{aj} e_j)|(\sum_{r=1}^n A_{br} e_r) = \sum_{j=1}^n A_{aj}A_{bj} = 0\]

	Expanding the full product, we have:
	\[w_1...w_k = (\sum_{j_1=1}^n A_{1{j_1}}e_{j_1})...(\sum_{j_k=1}^n A_{k{j_k}}e_{j_k})\]

	We can index the terms in the above product as follows: at each of the $k$ factors choose one of the terms in the sum (i.e. $A_{it} e_t$), construct a sequence $t(i)$ out of subsequent choices.

	The terms with grades lesser than $k$ (w.r.t. the orthonormal basis) are those terms for which at least one choice is repeated, we can use this to index all possible such terms: given a choice of $(q, m) \in \{1..k\}^2$ where the repetition ought to occur, we are free to choose any $c \in \{1..n\}$ for which $t(q) = t(m) = c$, moreover we are free to vary all other terms independently so we must sum over all subsequences $F(q,m) \equiv \{f(i): \{1..k\}\setminus\{q,m\} \to \{1..n\}\}$. 

	Thus, we may collect all lower grade terms in the following sum:
	\begin{align*}
		&\sum_{(q,m)} \sum_{f \in F} \sum_{c=1}^n (-1)^{\mg{m-q-1}}(A_{1f(1)}...A_{qc}A_{mc}...A_{kf(k)})e_{f(1)}...e_{f(k)} \\
	&= \sum_{(q,m)} \sum_{f \in F} (-1)^{\mg{m-q-1}}(A_{1f(1)}...A_{kf(k)})e_{f(1)}...e_{f(k)} \sum_{c=1}^n (A_{qc}A_{mc}) = 0
	\end{align*}
	
	So all lower-grade terms vanish. That the total product is non-zero simply follows from the fact that the algebra would otherwise permit a frame of dimension smaller than $2^n$.
\end{proof}

\begin{definition}
	The \textbf{pseudoscalar} of a geometric algebra $\G(\V)$ is the highest grade element in its frame. It is unique up to scalar multiplication and including permutation of factors.
	The square of a pseudoscalar is a scalar: by convention, it is normalized such that it squares to $1$ or $-1$ (defining the orientation of the frame).
\end{definition}


The distinction between basis and frame is very important: a frame is to a basis, as a geometric algebra is to its generating vector space.
For finite-dimensional spaces we can always produce an orthonormal basis by the Gram-Schmidt process, so we will henceforth assume all bases to be orthonormal unless otherwise specified.

It is then clear that for every inner product space, we may construct its Geometric Algebra by the following recipe:
\begin{enumerate}
	\item construct an orthonormal basis for the inner-product space: this will be the basis of the algebra
	\item construct the frame of the algebra by taking products of basis vectors
	\item generate the rest of the algebra as a vector space with the frame as its linear basis
\end{enumerate}

% With the above terminology we can also straightforwardly prove the Universal Property of a Geometric Algebra.
% \begin{definition}[Universal Property of a Geometric Algebra]
	Let $\V$ be a finite-dimensional vector space over the reals, and let $j: \V \times \V \to \R$ be an inner product, then there exists unique geometric algebra $\G(\V)$ and canonical injection $i: \V \to \G(\V)$ such that the following diagram commutes:
	\textcolor{red}{NOT QUITE!}
\end{definition}

% \begin{definition}[Universal Property of a Geometric Algebra]
	Let $\V$ be a finite-dimensional vector space over the reals, and let $j: \V \times \V \to \R$ be an inner product, then there exists unique geometric algebra $\G(\V)$ and canonical injection $i: \V \to \G(\V)$ such that the following diagram commutes:
	\textcolor{red}{NOT QUITE!}
\end{definition}


A brief note on conventions: here on out, whenever left unspecified, the precedence of products is inner$\to$wedge$\to$geometric.

%\newpage
%
%\section{Construction: Bases and Frames}\label{s:bases-frames}
%
%A Geometric Algebra is only realized into a concrete algebra once an inner-product space is specified. It is then straightforward to construct the whole algebra given a basis for the inner-product space. But first, we need some specific terminology.


%\newpage

\section{Products}\label{s:products}
We will now go on to consider general expressions and properties of products between arbitrary multivectors.
First off, we generalize the definition of the inner and outer products to homogeneous multivectors.
\begin{definition}\label{d:inner-product1}
	The inner product $A_r|B_s$ between two homogeneous multivectors $A_r$ and $B_s$ is the lowest-possible grade term in their product. We will later see that this is the $\mg{r-s}$-grade term; if the product does not have such a term, then we say that the inner product is 0.
\end{definition}

\begin{definition}\label{d:outer-product1}
	The outer product $A_r \wedge B_s$ between two homogeneous multivectors $A_r$ and $B_s$ is the highest-possible grade term in their product. We will later see that this is the $\mg{r+s}$-grade term; if the product does not have such a term, then we say that the outer product is 0.
\end{definition}


We now prove the following lemma regarding the product of a vector with a homogenous multivector.
\begin{lemma}\label{l:v-mv-product}
	The inner and outer products of a vector with a homogeneous multivector have the following expressions:
	\begin{align*}
		& a | A_r = \g{aA_r}{r-1} = \frac{1}{2}(aA_r-(-1)^rA_ra) \\
		& a \wedge A_r = \g{aA_r}{r+1} = \frac{1}{2}(aA_r+(-1)^rA_ra)
	\end{align*}
\end{lemma}

\begin{proof}
	We shall assume that $A_r$ is an $r$-blade: the case of a homogeneous multivector follows directly using distributivity of the geometric product (since an $r$-grade homogeneous multivector is a sum of $r$-blades).

	Recall the definition of the inner product between two vectors (def. \ref{d:inner-product1})
	\[a | b = \frac{1}{2}(ab + ba)\]
	We can reverse it to obtain:
	\[ab = 2 a | b - ba\]
	Repeated application of the above allows us to permute indices in a product, as follows:
	\begin{align*}
		aA_r = aa_1a_2 \ldots a_r = &2a|a_1a_2 \ldots a_r - a_1aa_2 \ldots a_r \\
		= &2a|a_1a_2 \ldots a_r - 2a|a_2a_1 \ldots a_r + a_1aa_2 \ldots a_r \\
		= &\ldots \\
		= &2 \sum_{k=1}^r (-1)^{k+1}a|a_k a_1 \ldots \check{a_k} \ldots a_r + (-1)^ra_1a_2 \ldots a_ra \\
		= &\sum_{k=1}^r (-1)^{k+1}a|a_k a_1 \ldots \check{a_k} \ldots a_r + \sum_{k=1}^r (-1)^{k+1}a|a_k a_1 \ldots \check{a_k} \ldots a_r \\
		&+ (-1)^ra_1a_2 \ldots a_ra
	\end{align*}

	Notice that the first term above has grade $r-1$: we will demonstrate that this is indeed the lowest grade term. For now, we preemptively denote it $a | A_r$.
	
	Let us rewrite the sum using invertibility of vectors (th. \ref{l:invertibility}):
	\begin{align*}
		a | A_r &= \sum_{k=1}^r (-1)^{k+1}a|a_k a_1 \ldots \check{a_k} \ldots a_r\\
				&= \sum_{k=1}^r (-1)^{k+1}a|a_k a_k^{-1}a_k a_1 \ldots \check{a_k} \ldots a_r\\
				&= \sum_{k=1}^r a|a_k a_k^{-1} A_r 
	\end{align*}

	Substracting the above from $aA_r$ and factoring:
	\[aA_r - a | A_r = (a - \sum_{k=1}^r a|a_k a_k^{-1})A_r \equiv bA_r\]
	
	Since by construction, $b | a_k = 0 \forall k \in \{1, \ldots r\}$, it follows by Corollary \ref{c:orthogonality} that the above is a product of $r+1$ anticommuting vectors and thus has grade $r+1$ by Definition \ref{d:k-blades}. We are justified in writing:

	\begin{align*}
		aA_r &= a | A_r + a \wedge A_r \\
		a | A_r &= \sum_{k=1}^r (-1)^{k+1}a|a_k a_1 \ldots \check{a_k} \ldots a_r \\
		a \wedge A_r &= a | A_r + (-1)^rA_ra
	\end{align*}

	From which the lemma follows straightforwardly by substituting the third expression into the first:
	\begin{align*}
		aA_r &= 2a | A_r + (-1)^rA_ra \Rightarrow a | A_r = \frac{1}{2} (aA_r - (-1)^rA_ra) \\
		aA_r &= 2a \wedge A_r - (-1)^rA_ra \Rightarrow a \wedge A_r = \frac{1}{2} (aA_r + (-1)^rA_ra)
	\end{align*}
\end{proof}


Using the above lemma, we can prove the following important property of the geometric product between homogeneous multivectors.
\begin{theorem}[Product of Homogeneous Multivectors]\label{t:homog-product}
	The product of homogeneous multivectors $A_r, B_s$ can be decomposed as follows:
	\[A_rB_s = \sum_{k=0}^{\min\{r,s\}} \g{A_rB_s}{\mg{s-r}+2k}\]
\end{theorem}

\begin{proof}
	We prove this by induction on $r \leq s$ when $A_r$ and $B_s$ are simple $r$- and $s$-vectors respectively.

	The case $r=1,\ s=1$ is true by Definition \ref{t:geometric-product}.
	The case $r=1,\ s>1$ is true by Lemma \ref{l:v-mv-product}.
	Assume the expression holds for $r = q,\ s > r$, we show that it holds for $r+1$:
	\begin{alignat*}{2}
		A_{r+1}B_s &=\ &&a_{r+1}A_rB_s = a_{r+1}\sum_{k=0}^r \g{A_rB_s}{s-r+2k} \\
				   &= &&\sum_{k=0}^r a_{r+1}\g{A_rB_s}{s-r+2k} \\
				   &= &&\sum_{k=0}^r \left[a_{r+1} | \g{A_rB_s}{s-r+2k} + a_{r+1} \wedge \g{A_rB_s}{s-r+2k}\right]\\
				   &=\ &&a_{r+1} | \g{A_rB_s}{s-r} \\
				   &  &&+ \sum_{k=1}^r \left[a_{r+1} | \g{A_rB_s}{s-r+2k} + a_{r+1} \wedge \g{A_rB_s}{s-r+2(k-1)}\right]\\
				   &  &&+ a_{r+1}\wedge \g{A_rB_s}{s+r}
	\end{alignat*}
	where in the last step, we have grouped terms together by grade:
	\begin{alignat*}{2}
		&\g{A_{r+1}B_s}{s-(r+1)} &&\equiv a_{r+1} | \g{A_rB_s}{s-r} \\
		&\g{A_{r+1}B_s}{s-(r+1)+2k} &&\equiv a | \g{A_rB_s}{s-r+2k} + a \wedge \g{A_rB_s}{s-r+2(k-1)} \\
		&\g{A_{r+1}B_s}{r+1+s} &&\equiv a \wedge \g{A_rB_s}{s+r}
	\end{alignat*}

	The case where $s \leq r$ follows by induction on $s$ (the same argument as above).
	The general case for homogeneous multivectors follows by distributivity of the geometric product.
\end{proof}

The above proof is due to \textit{Chisolm} \cite[p. 20-21]{ga-chisolm}


As a corollary, we obtain our sought-after result.
\begin{corollary}[Even Subalgebras]\label{c:even-subalgebras}
	The set of even-grade elements of a geometric algebra constitutes a subalgebra.
\end{corollary}


It also follows that the inner and outer products can be naturally generalized to arbitrary multivectors as well-defined bilinear operations on $\G(\V)$.
\begin{corollary}
	(Generalized Inner Product)
	
	On homogeneous multivectors:
	\[A_r | B_s \equiv \g{A_rB_s}{\mg{s-r}}\]
	%represents the orthogonal complement of the smaller space in the larger space (left\right)
	
	On arbitrary multivectors:
	\[A | B \equiv \sum_{r=0}^{\gr{A}}\sum_{s=r}^{\gr{B}} \g{A}{r} | \g{B}{s}\]
\end{corollary}

\begin{corollary}
	(Generalized Outer Product)

	On homogeneous multivectors:
	\[A_r \wedge B_s \equiv \g{A_rB_s}{s+r}\]
	%represents the orthogonal complement of the smaller space in the larger space (left\right)
	
	On arbitrary multivectors:
	\[A \wedge B \equiv \sum_{r=0}^{\gr{A}}\sum_{s=r}^{\gr{B}} \g{A}{r} \wedge \g{B}{s}\]
\end{corollary}


