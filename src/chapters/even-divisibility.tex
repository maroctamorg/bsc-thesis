\section{Even Subalgebras}
We start off with a discussion on the construction of even subalgebras and introduce some notation and results that will be of use in the next section.

An even subalgebra has its own basis which is distinct from that of the original algebra; this is made explicit in the following self-evident lemma:
\begin{lemma}
	Let $\G$ be a geometric algebra with basis $\{e_1, e_2, \ldots e_n\}$.
	Its even subalgebra $\G_+$ has basis $\{e_1e_2, \ldots e_1e_n, e_2e_3 \ldots e_{n-1}e_n\}$; we will make use of the following shortened notation: $e_ie_j = e_{ij}$.

	The basis bivectors have the following basic properties:
	\begin{align*}
		e_{ii} &= 0 \\
		e_{ij} &= -e_{ji} \\
		{e_{ij}}^2 &= -1 \\
	\end{align*}
	Moreover, it follows from straightforward algebraic manipulations that the product of two unit bivectors anti-commutes if they share a common index and commutes otherwise.

\end{lemma}

We will make use of the above properties in computation, without explicit mention.

An even subalgebra is constructed from its basis precisely in the same manner as a geometric algebra.

\newpage

\section{Euclidean Geometric Algebras}
We will follow the approach laid out in the previous section in order to construct the Geometric Algebra of the 1, 2 and 3 dimensional Euclidean spaces and consider their even subalgebras.
The main result of this chapter will be our proof that these are isomorphic to the well-known scalar algerbras $\R, \C \mathrm{\ and\ } \HM$.

\subsection{$\G(\E_1)\ \mathrm{and}\ \G_+(\E_1)$}

$\G(\E_1)$  is trivial: the inner product is simply given by the magnitude of the standard product (i.e. $a | b = \mg{ab}$); and its frame is just $\{1, e_1\}$.
It follows that its even subalgebra has frame $\{1\}$ and thus corresponds to $\R$.

\subsection{$\G(\E_2)\ \mathrm{and}\ \G_+(\E_2)$}

$\E_2$ is the standard euclidean plane. We denote its orthonormal basis $\{\bv{e}_1, \bv{e}_2\}$ and work out its multiplication table with respect to the geometric product:
\begin{table}[h!]
	\centering
	\begin{tabular}{|c|c|c|c|c|}
		\hline
		$\cdot$ & 1 & $\bv{e}_1$ & $\bv{e}_2$ & $\bv{e}_1\bv{e}_2$ \\
		\hline
		1 & 1 & $\bv{e}_1$ & $\bv{e}_2$ & $\bv{e}_1\bv{e}_2$ \\
		\hline
		$\bv{e}_1$ & $\bv{e}_1$ & 1 & $\bv{e}_1\bv{e}_2$ & $\bv{e}_2$ \\
		\hline
		$\bv{e}_2$ & $\bv{e}_2$ & $-\bv{e}_1\bv{e}_2$ & 1 & $-\bv{e}_1$ \\
		\hline
		$\bv{e}_1\bv{e}_2$ & $\bv{e}_1\bv{e}_2$ & $-\bv{e}_2$ & $\bv{e}_1$ & -1 \\
		\hline
	\end{tabular}
	\caption{Multiplication table of $\G(\E_2)$}
	\label{ta:mt2a}
\end{table}


The frame of $\G(\E_2)$ is simply $\{1, \bv{e}_1, \bv{e_2}, \bv{e}_1\bv{e_2}\}$, and the even subalgebra $\G_+(\E_2)$ is thus generated by $\{1, \bv{e}_1\bv{e_2} \}$.
It is apparent that the mapping $\bv{e}_1\bv{e_2} \leftrightarrow i$ induces an isomorphism $j : \G(\E_2) \leftrightarrow \C$ by comparing multiplication tables:
\begin{table}[h!]
\parbox{.45\linewidth}{
	\raggedleft
	\begin{tabular}{|c|c|c|}
		\hline
		$\cdot$ & 1 & $\bv{e}_1\bv{e}_2$ \\
		\hline
		1 & 1 & $\bv{e}_1\bv{e}_2$ \\
		\hline
		$\bv{e}_1\bv{e}_2$ & $\bv{e}_1\bv{e}_2$ & -1 \\
		\hline
	\end{tabular}
}
\hskip .1\linewidth
\parbox{.45\linewidth}{
	\raggedright
	\begin{tabular}{|c|c|c|}
		\hline
		$\cdot$ & 1 & $i$ \\
		\hline
		1 & 1 & $i$ \\
		\hline
		$i$ & $i$ & -1 \\
		\hline
	\end{tabular}
}
\caption{Multiplication tables of $\G_+(\E_2)$ and $\C$}
\label{ta:mt2b}
\end{table}


\subsection{$\G(\E_3)\ \mathrm{and}\ \G_+(\E_3)$}
$\E_3$ is the standard euclidean space. We denote its orthonormal basis $\{\bv{e}_1, \bv{e}_2\, \bv{e}_3\}$.

The frame of $\G(\E_3)$ is simply $\{1, \bv{e}_1, \bv{e}_2, \bv{e}_3, \bv{e}_1\bv{e}_2, \bv{e}_2\bv{e}_3, \bv{e}_1\bv{e}_3, \bv{e}_1\bv{e}_2\bv{e}_3\}$; the even subalgebra $\G_+(\E_3)$ is thus generated by the set $\{1, \bv{e}_1\bv{e}_2, \bv{e}_2\bv{e}_3, \bv{e}_1\bv{e}_3\}$; by observing its multiplication table (Table \ref{ta:mt3}), it becomes clear that the mapping
\begin{align*}
	&\bv{e}_1\bv{e}_2 \leftrightarrow i \\
	&\bv{e}_2\bv{e}_3 \leftrightarrow j \\
	&\bv{e}_1\bv{e}_3 \leftrightarrow k 
\end{align*}
induces an isomorphism $j : \G(\E_3) \leftrightarrow \HM$.

\begin{table}[h!]
\parbox{.45\linewidth}{
    \begin{tabular}{|c|c|c|c|c|}
    \hline
    $\cdot$ & 1 & $e_{12}$ & $e_{23}$ & $e_{13}$ \\
    \hline
    1 & 1 & $e_{12}$ & $e_{23}$ & $e_{13}$ \\ 
    \hline
    $e_{12}$ & $e_{12}$ & -1 & $e_{13}$ & $-e_{23}$ \\ 
    \hline
    $e_{23}$ & $e_{23}$ & $-e_{13}$ & -1 & $e_{12}$\\
    \hline
    $e_{13}$ & $e_{13}$ & $e_{23}$ & $-e_{12}$ & -1\\
    \hline
    \end{tabular}
}
\hskip .1\linewidth
\parbox{.45\linewidth}{
    \begin{tabular}{|c|c|c|c|c|}
    \hline
    $\cdot$ & 1 & $i$ & $j$ & $k$ \\
    \hline
    1 & 1 & $i$ & $j$ & $k$ \\ 
    \hline
    $i$ & $i$ & -1 & $k$ & $-j$ \\ 
    \hline
    $j$ & $j$ & $-k$ & -1 & $i$\\
    \hline
    $k$ & $k$ & $j$ & $-i$ & -1\\
    \hline
    \end{tabular}
}
\caption{Multiplication tables of $\G_+(\E_3)$ and $\HM$}
\label{ta:mt3}
\end{table}


We conclude with the following lemma which shows that for $n > 3$, the even subalgebras are no longer division algebras.

\begin{lemma}[Non-divisibility of higher-dimensional even euclidean subalgebras]\label{l:nondivisibility}
	Let $\G_+(\E_n)$ denote the even geometric subalgebra of the $n$-dimensional Euclidean space.
	$\G_+(\E_n)$ is a division algebra if and only if $n \leq 3$.
\end{lemma}

\begin{proof}
	Consider the following element
	\[u = e_{12}e_{13}e_{(n-1)n}\]
	We show that it squares to $1$:
	\begin{align*}
		u^2 = &e_{12}e_{13}e_{(n-1)n}e_{12}e_{13}e_{(n-1)n}\\
			= &e_{12}e_{12}e_{13}e_{13}e_{(n-1)n}e_{(n-1)n}\\
			= &-1
	\end{align*}


