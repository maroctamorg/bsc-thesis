\section{Historical Background}

\section{Aim and Content}

We start with an overview of the field, aiming to address the following questions: what is a Geometric Algebra? How is it defined? What are its main components? How do different Geometric Algebras relate to one another? 

I should preface this discussion by noting that Geometric Algebra - as the mathematical project proposed by \textit{David Hestenes} and \textit{Garret Sobczyk} in \cite{ga-origin} - is a relatively new subject: as such, there seems to be no consolidated consensus as to the precise manner in which it ought to be presented. This paper will follow most closely the original approach in \cite{ga-origin}; nonetheless, I have taken the liberty to formulate and present some of these concepts in a slightly different manner (though only superficially so). In particular, this applies to the choice of axioms (the equivalence of which will be demonstrated), inspired by \textit{Eric Chisolm}'s take on the topic \cite{ga-chisolm}. I have also taken care to discuss a couple of points which I think have been overlooked in other treatments of the topic: mainly, the \textbf{Universal Geometric Algebra} as a \textbf{template\textcolor{red}{[CATEGORY]}} and the relationship of \textbf{Geometric Algebra} to \textbf{Clifford Algebra}.

\textbf{Geometric Algebra} concerns itself with the construction of a family\textcolor{red}{[CATEGORY]} of abstract algebras which adequately extend the arithmetic of the real numbers to higher-dimensional, coordinate-free settings, under the guidance of geometric intuition. In a \textbf{Geometric Algebra} elements are fully characterised by three (geometric) properties: \textbf{magnitude}, \textbf{direction} and \textbf{orientation}. In practice, the theory identifies linear spaces with Euclidean geometric primitives, and in doing so equips any (finite-dimensional) linear space with a powerful algebra whose elements are the subspaces themselves.

\textcolor{red}{This is reminiscent of the Greek view of mathematics in terms of geometric constructions: numbers as segments, ...}

In his original paper, \textit{Hestenes} proposes that Geometric Algebra (and Calculus) can and should provide a unified framework for mathematical physics. In his own words: \textit{``Our long-range aim is to see Geometric Calculus established as a unified system for handling linear and multilinear algebra, multivariable calculus, complex variable theory, differential geometry and other subjects with geometric content."} (p. ix)\cite{ga-origin}.

The main

Our main focus in this paper, will be in the mathematical development of the theory and its usefulness in the field of Abstract Algebra: specifically, in the classification of Scalar Algebras.
