\section{Historical Background}
Geometric Algebra has its roots in the long-running pursuit of a mathematical framework for the description of physical space. Though the concept of vectors is ancient, in this section we shall focus on the developments starting in the 19th century.

In the early 1800s, \textbf{Grassmann} was the first to formulate the notions of `modern' linear algebra (vector spaces, bases, inner product and orthogonality) \cite{grassmann}; his development of exterior (or as he called it, extended) algebra laid the key theoretical groundwork for Clifford's Geometric Algebra. \cite[p.28-29]{ga-foundations}.
%\begin{align*}
%	\bv{e_i} | \bv{e_j} &= \delta_{ij}\\
%	\bv{e_i}\wedge \bv{e_j} &= -\bv{e_i}\wedge \bv{e_j}\\
%\end{align*}

Around the same time, \textbf{Hamilton} extends the 2-dimensional algebra of the complex numbers into the 4-dimensional algebra of the quaternions, aiming at a formal vector algebra for $\mathbb{R}^3$.
%\begin{align*}
%    &q = q_0 + q_1\boldsymbol{i} + q_2\boldsymbol{j} + q_3\boldsymbol{k}\\
%    &i^2 = j^2 = k^2 = ijk = -1\\
%    &ij = k, ki = j, jk = i
%\end{align*}
The quaternion algebra - though a division algebra and initially adopted by Maxwell in his formulation of electrodynamics - faced some pushback and was eventually mostly supplanted by Gibbs' vector algebra (especially in physical applications) and became relegated to specific applications. Some would argue that the shortcomings of the quaternions are grounded in the fact that quaternions provide a representation for the algebra of rotations in three-dimensions, and are not natural representations of cartesian vectors in $\R^3$ (a major shortcoming being that quaternions do not generalize to higher dimensions) \cite{ga-history}.

\textbf{Gibbs} is responsible for the standard vector algebra that is most widely used today: his insight was to separate the product of quaternion 'vectors' into dot and cross products and formally replace the imaginary units with unit vectors that square to $+1$.
%\begin{align*}
%	vu &= -\sum v_iu_i + \sum_{i\to j\to k}(v_iu_j - v_ju_i)\boldsymbol{k} \\
%	&\equiv -v | u + v \times u
%\end{align*}

The development of standard vector calculus (e.g. $\nabla, \nabla \cdot, \nabla \times$), applied extensively in Electrodynamics by \textbf{Heaviside}, lead to the widespread adoption of the formalism.

Though it became the adopted formalism, \textbf{Gibbs}' vector algebra has its own deficiencies: it requires two separate vectors product, lacks a division operation and its cross product does not generalize to higher-dimensions.

\textbf{Clifford}'s  Geometric Algebra - introduced in the second half of the 19th century - aimed to supersede both formalisms by providing an algebra that could be generalized to higher dimensions, and adequately and comprehensively describe both vectors and transformations in space; it extends Grassmann's work, incorporating Hamilton's quaternions into a generalizable algebraic system for vectors, based on the geometric product.
%\begin{align*}
%    &uv = u | v + u \wedge v\\
%    &V = \sum \langle V\rangle_i\\
%    &\g{V}{r} = \sum v_i\bv{e_{i_1}}\wedge ... \wedge \bv{e_{i_r}}
%\end{align*}

Though \textbf{Clifford}'s work was mostly neglected in his time, it finds modern applications in a few areas of mathematics including differential geometry and mathematical physics (especially with regards to spinors). A mathematical project which aims to establish Geometric Algebra as a comprehensive, unifying framework for a variety of mathematical theories and tools related to the description of space, has also been put forth recently by \textbf{David Hestenes} \cite{ga-origin}.% We will be discussing the modern formalism in this thesis.

%\newpage

\section{Content and Aim}

Our main focus in this paper, will be in the axiomatic approach to Geometric Algebra: we will be proving basic properties and theorems from the axioms and use these towards constructing the Euclidean Geometric Algebras and their Even Subalgebras in 1, 2 and 3 dimensions, demonstrating that the latter are actually isomorphic to the Real, Complex and Quaternion algebras.

In section \ref{s:axioms-definitions}, we provide an axiomatic formulation of Geometric Algebra as well as present the main objects and their definitions. The most important results in this section are Theorem \ref{t:geometric-product} which separates the geometric product of two vectors into an inner and an outer product and Theorem \ref{t:frame} which affirms that every multivector has a unique decomposition in terms of more basic objects called $k$-blades. We also make some remarks about the construction of a Geometric Algebra for a given Inner Product Space.

Section \ref{s:products} contains a few definitions and a minor lemma leading up to Theorem \ref{t:homog-product} which characterises the geometric product between homogeneous multivectors from which we obtain the main result of Chapter \ref{ch:ga} in the form of Corollary \ref{c:even-subalgebras} which states that the set of even-graded elements constitutes a subalgebra.

In Chapter \ref{ch:euclidean-even}, we construct the Geometric Algebras of the 1, 2 and 3-dimensional Euclidean Spaces and prove that their Even Subalgebras are indeed isomorphic to the scalar algebras. We conclude by showing that the higher-dimensional even subalgebras are non-divisible, as required by the Frobenius Classification of the Real Associative Finite-Dimensional Division Algebras.

%\newpage

\section{Geometric Algebra as a Project}
Geometric Algebra - as the mathematical project proposed by \textit{David Hestenes} and \textit{Garret Sobczyk} in \cite{ga-origin} - is a relatively new subject: as such, there seems to be no consolidated consensus as to the precise manner in which it ought to be presented. This thesis will diverge from the approach in \cite{ga-origin} in a few important ways; in particular, this applies to the choice of axioms - closer to Vaz and da Rocha's take on the topic \cite{clifford-algebra} and our limitation to finite-dimensional inner product spaces.

\textbf{Geometric Algebra} concerns itself with the construction of a collection of abstract algebras which adequately extend the arithmetic of the real numbers to higher-dimensional, coordinate-free settings, under the guidance of geometric intuition. Intuitively, the theory identifies linear spaces with Euclidean geometric primitives (think points, segments, parallelograms...), and in doing so equips any (finite-dimensional) linear space with a powerful algebra whose elements are the subspaces themselves.

This is reminiscent of the Greek pre-algebraic view of mathematics, whereby arithmetic was intrinsically tied to geometric constructions.
Perhaps for this very reason, \textbf{Geometric Algebra} has had made significant progress in its express goal of providing a unified framework for mathematical physics and applied mathematics. In his book, \textit{Hestenes} details this more specifically \cite[p. ix]{ga-origin}:

\begin{quote}
	``Our long-range aim is to see Geometric Calculus established as a unified system for handling linear and multilinear algebra, multivariable calculus, complex variable theory, differential geometry and other subjects with geometric content.''
\end{quote}
