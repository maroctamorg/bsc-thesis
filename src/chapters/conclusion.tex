In this paper, we have developed the fundamentals of Geometric Algebra.

Starting from a formal set of axioms, we have thoroughly derived the general properties of the algebra, its elements and its products.

We have observed that the algebra does indeed encompass the standard Gibbs' algebra, the algebra of the Complex Numbers and that of the Quaternions. All while having the marked advantage of invertibility with regards to vectors, a natural generalization to higher dimensions, as well as the capability to encode Euclidean geometric primitives through the exterior product.

Geometric Algebra - and its extension to the domain of analysis, Geometric Calculus - have many promising applications that unfortunately were outside the scope of this thesis: linear and multilinear algebra can be expressed in the language of frames and outermorphisms; differential geometry can be formulated in the framework of vector manifolds and the geometric derivative with the promise of coordinate-free computations; the even subalgebras of Euclidean geometric spaces can be shown to account for the theory of Spinors; advances have even been made in the theory of Lie Algebras and Groups.

\textbf{Geometric Algebra} postulates that geometry is a fundamental guide towards meaningful mathematical pursuits even in abstract settings. It should be no surprise that such a theory would be so adept at describing a wide range of geometric phenomena. As \textit{Hestenes} puts it (p. xii)\cite{ga-origin}:
\begin{quote}
	``Geometry without algebra is dumb! Algebra without geometry is blind!''
\end{quote}


